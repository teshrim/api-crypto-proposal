\documentclass[10pt]{article}
%\setlength{\topmargin}{-0.5in}
\setlength{\textwidth}{6.25in}
\setlength{\textheight}{9in}
\setlength{\oddsidemargin}{-0.125in}
\setlength{\evensidemargin}{-0.125in}

\newenvironment{packeditemize}{
\begin{list}{$\bullet$}{
\setlength{\labelwidth}{8pt}
\setlength{\itemsep}{0pt}
\setlength{\leftmargin}{\labelwidth}
\addtolength{\leftmargin}{\labelsep}
\setlength{\parindent}{0pt}
\setlength{\listparindent}{\parindent}
\setlength{\parsep}{0pt}
\setlength{\topsep}{3pt}}}{\end{list}}

%\newcommand{\bnm}{\begin{newmath}}
\newcommand{\enm}{\end{newmath}}

\newcommand{\bea}{\begin{eqnarray*}}
\newcommand{\eea}{\end{eqnarray*}}



\newcommand{\bne}{\begin{newequation}}
\newcommand{\ene}{\end{newequation}}

\newenvironment{newmath}{\begin{displaymath}%
\setlength{\abovedisplayskip}{4pt}%
\setlength{\belowdisplayskip}{4pt}%
\setlength{\abovedisplayshortskip}{6pt}%
\setlength{\belowdisplayshortskip}{6pt} }{\end{displaymath}}

\newenvironment{neweqnarrays}{\begin{eqnarray*}%
\setlength{\abovedisplayskip}{-4pt}%
\setlength{\belowdisplayskip}{-4pt}%
\setlength{\abovedisplayshortskip}{-4pt}%
\setlength{\belowdisplayshortskip}{-4pt}%
\setlength{\jot}{-0.4in} }{\end{eqnarray*}}

\newenvironment{newequation}{\begin{equation}%
\setlength{\abovedisplayskip}{4pt}%
\setlength{\belowdisplayskip}{4pt}%
\setlength{\abovedisplayshortskip}{6pt}%
\setlength{\belowdisplayshortskip}{6pt} }{\end{equation}}


\newcommand{\Adv}{\mathbf{Adv}}
\newcommand{\AdvMI}[1]{\Adv^\mathrm{mi}_{#1}}
\newcommand{\AdvHEDIST}[1]{\Adv^\mathrm{dist}_{#1}}
\newcommand{\AdvMR}[1]{\Adv^\mathrm{mr}_{#1}}
\newcommand{\AdvMRCCA}[1]{\Adv^\mathrm{mr\textnormal{-}cca}_{#1}}
\newcommand{\AdvKR}[1]{\Adv^\mathrm{kr}_{#1}}
\newcommand{\AdvSAMPRAT}[1]{\Adv^\mathrm{dte\textnormal{-}ratio}_{#1}}
\newcommand{\AdvSAMPIND}[1]{\Adv^\mathrm{dte}_{#1}}
%\newcommand{\AdvDTE}[1]{\Adv^\mathrm{dte}_{#1}}

\newcommand{\decOracle}{\textbf{Dec}}

\newcommand{\negsmidge}{{\hspace{-0.1ex}}}
\newcommand{\cdotsm}{\negsmidge\negsmidge\negsmidge\cdot\negsmidge\negsmidge\negsmidge}

\def\suchthatt{\: :\:}

\newcommand{\Prob}[1]{{\Pr\left[\,{#1}\,\right]}}
\newcommand{\probb}[2]{{\Pr}_{#1}\left[\,{#2}\,\right]}
\newcommand{\Probb}[2]{\Pr[#1]}
\newcommand{\CondProb}[2]{{\Pr}\left[\: #1\:\left|\right.\:#2\:\right]}
\newcommand{\CondProbb}[2]{\Pr[#1|#2]}
\newcommand{\ProbExp}[2]{{\Pr}\left[\: #1\:\suchthatt\:#2\:\right]}
\newcommand{\Ex}[1]{{\textnormal{E}\left[\,{#1}\,\right]}}
\newcommand{\Exx}{{\textnormal{E}}}

\newcommand{\true}{\textsf{true}}
\newcommand{\false}{\textsf{false}}




\newcommand{\getsr}{{\:{\leftarrow{\hspace*{-3pt}\raisebox{.75pt}{$\scriptscriptstyle\$$}}}\:}}
\newcommand{\getm}{{\:\leftarrow_{\mdist}\:}}
\newcommand{\getmcand}{{\:\leftarrow_{\mcanddist}\:}}
\newcommand{\getd}{{\:\leftarrow_{\ddist}\:}}
%\newcommand{\getm}{{\:{\leftarrow{\hspace*{-3pt}\raisebox{.75pt}{$\scriptscriptstyle \mdist$}}}\:}}
\newcommand{\getk}{{\:\leftarrow_{\kdist}\:}}
%\newcommand{\getk}{{\:{\leftarrow{\hspace*{-3pt}\raisebox{.75pt}{$\scriptscriptstyle \kdist$}}}\:}}
\newcommand{\getp}{{\:\leftarrow_{p}\:}}



\newcommand{\gamesfontsize}{\small}
\newcommand{\fpage}[2]{\framebox{\begin{minipage}[t]{#1\textwidth}\setstretch{1.1}\gamesfontsize  #2 \end{minipage}}}

\newcommand{\hpages}[3]{\begin{tabular}{cc}\begin{minipage}[t]{#1\textwidth} #2 \end{minipage} & \begin{minipage}[t]{#1\textwidth} #3 \end{minipage}\end{tabular}}


\newcommand{\hfpages}[3]{\hfpagess{#1}{#1}{#2}{#3}}
\newcommand{\hfpagess}[4]{
        \begin{tabular}{c@{\hspace*{.5em}}c}
        \framebox{\begin{minipage}[t]{#1\textwidth}\setstretch{1.1}\gamesfontsize #3 \end{minipage}}
        &
        \framebox{\begin{minipage}[t]{#2\textwidth}\setstretch{1.1}\gamesfontsize #4 \end{minipage}}
        \end{tabular}
    }
\newcommand{\hfpagesss}[6]{
        \begin{tabular}{c@{\hspace*{.5em}}c@{\hspace*{.5em}}c}
        \framebox{\begin{minipage}[t]{#1\textwidth}\setstretch{1.1}\gamesfontsize #4 \end{minipage}}
        &
        \framebox{\begin{minipage}[t]{#2\textwidth}\setstretch{1.1}\gamesfontsize #5 \end{minipage}}
        &
        \framebox{\begin{minipage}[t]{#3\textwidth}\setstretch{1.1}\gamesfontsize #6 \end{minipage}}
        \end{tabular}
    }
\newcommand{\hfpagessss}[8]{
        \begin{tabular}{c@{\hspace*{.5em}}c@{\hspace*{.5em}}c@{\hspace*{.5em}}c}
        \framebox{\begin{minipage}[t]{#1\textwidth}\setstretch{1.1}\gamesfontsize #5 \end{minipage}}
        &
        \framebox{\begin{minipage}[t]{#2\textwidth}\setstretch{1.1}\gamesfontsize #6 \end{minipage}}
        &
        \framebox{\begin{minipage}[t]{#3\textwidth}\setstretch{1.1}\gamesfontsize #7 \end{minipage}}
        &
        \framebox{\begin{minipage}[t]{#4\textwidth}\setstretch{1.1}\gamesfontsize #8 \end{minipage}}
        \end{tabular}
    }

\newcommand{\vecw}{\mathbf{w}}
\newcommand{\R}{\mathbb{R}}
\newcommand{\N}{\mathbb{N}}
\newcommand{\Z}{\mathbb{Z}}
\newcommand{\load}{L}
\newcommand{\coll}{\mathsf{Coll}}
\newcommand{\nocoll}{\overline{\mathsf{Coll}}}


\newcommand{\Img}{\textsf{Img}}


\newcommand{\myind}{\hspace*{1em}}
\newcommand{\thh}{^{\textit{th}}} % th
\newcommand{\concat}{\,\|\,}
\newcommand{\dotdot}{..}
\newcommand{\emptystr}{\varepsilon}


\newcommand{\round}{\textsf{round}}

\newcommand{\alphabar}{\overline{\alpha}}
\newcommand{\numbinsbar}{\overline{b}}
\newcommand{\numballs}{a}
\newcommand{\numbins}{b}


\def \calR {{\mathcal{R}}}
\def \gen {{\textsf{gen}}}
\def \rejsam {{\textsf{RejSam}}}
\def \xspace {{\cal{X}}}
\def \yspace {{\cal{Y}}}
\def \xdist {{p_x}}
\def \ydist {{p_y}}
\def \mspace {{\cal{M}}}
\def \mspacecand {{\cal{N}}}
\def \mspacebot {{\cal{M}_\bot}}
\def \sspace {{\cal{S}}}
\def \slen {{s}}
\def \kspace {{\cal{K}}}
\def \kspacesize {{m}}
\def \mspacesize {{n}}
\def \kdict {D}
\def \dictsize {d}
\newcommand{\kdist}{p_k}
\newcommand{\mdist}{p_m}
\newcommand{\mcanddist}{g}
\newcommand{\ddist}{p_d}
\newcommand{\sdist}{p_s}
%\def \kdist {{\kappa}}
%\def \mdist {{\mu}}
%\def \ddist {{\delta}}
\def \pspace {{\cal{P}}}
\def \mpspace {{\cal{MP}}}
\def \cspace {{\cal{C}}}
\def \key {K}
\def \msg {M}
\def \secret {K}
\def \seed {S}
\def \ctxt {C}
\def \ctxtpart {C_2}
\def \DTE {{\textsf{DTE}}}
\newcommand{\genprime}{{\textsf{GenPrime}}}
\newcommand{\isprime}{{\textsf{IsPrime}}}
\newcommand{\divisible}{{\textsf{IsDiv}}}
\newcommand{\LeastLesserPrime}{{\textsf{PrevPrime}}}
\newcommand{\GetPrevDiv}{{\textsf{PrevPrimeDiv}}}
\def \encode {{\textsf{encode}}}
\newcommand{\DTEis}{{\textsf{IS-DTE}}}
\newcommand{\encodeis}{{\textsf{is-encode}}}
\newcommand{\decodeis}{{\textsf{is-decode}}}
\newcommand{\DTErej}{{\textsf{REJ-DTE}}}
\newcommand{\encoderej}{{\textsf{rej-encode}}}
\newcommand{\decoderej}{{\textsf{rej-decode}}}

%\newcommand{\encodeis}{{\textsf{encode}_{\textrm{is}}}}
%\newcommand{\decodeis}{{\textsf{decode}_{\textrm{is}}}}
\newcommand{\rep}{\textsf{rep}}
\newcommand{\isErr}{\epsilon_{\textnormal{is}}}
\newcommand{\incErr}{\epsilon_{\textnormal{inc}}}
\def \decode {{\textsf{decode}}}
\def \enc {{\textsf{enc}}}
\def \dec {{\textsf{dec}}}
\def \SEscheme {{\textsf{SE}}}
\def \HEscheme {{\textsf{HE}}}
\def \CTR {{\textsf{CTR}}}
\def \encHE {{\textsf{HEnc}}}
\def \HIDE {{\textsf{HiaL}}}
\def \encHIDE {{\textsf{HEnc}}}
\def \decHIDE {{\textsf{HDec}}}
\def \decHE {{\textsf{HDec}}}
\def \encHEt {{\textsf{HEnc2}}}
\def \decHEt {{\textsf{HDec2}}}
\def \rec {{\textsf{Rec}}}
\def \ssketch  {{\textsf{SS}}}
\def \dsrec {\tilde{{\textsf{Rec}}}}
\def \dsssketch  {\tilde{{\textsf{SS}}}}
\def \dist  {{\textsf{dist}}}


\newcommand{\oddnums}{\mathbb{O}}


\newcommand{\DTErsarej}{{\textsf{RSA-REJ-DTE}}}
\newcommand{\encodeRSAREJ}{{\textsf{rsa-rej-encode}}}
\newcommand{\decodeRSAREJ}{{\textsf{rsa-rej-decode}}}
\newcommand{\DTErsainc}{{\textsf{RSA-INC-DTE}}}
\newcommand{\encodeRSAINC}{{\textsf{rsa-inc-encode}}}
\newcommand{\decodeRSAINC}{{\textsf{rsa-inc-decode}}}
\newcommand{\DTEunf}{{\textsf{UNF-DTE}}}
\newcommand{\DTEnunf}{{\textsf{NUNF-DTE}}}
\newcommand{\DTErsassl}{{\textsf{RSA-SSL-DTE}}}
\newcommand{\encodeRSASSL}{{\textsf{rsa-ssl-encode}}}
\newcommand{\decodeRSASSL}{{\textsf{rsa-ssl-decode}}}



%\def \encHE {{\sf{enc}^{HE}}}
%\def \decHE {{\sf{dec}^{HE}}}
%\def \encHEt {{\sf{enc}^{HE2}}}
%\def \decHEt {{\sf{dec}^{HE2}}}
\def \idealHE {{\mathcal{HE}}}
\def \IEnc {{\mathbf{\rho}}}
\def \IDec {{\mathbf{\rho^{-1}}}}
\def \OEnc {{\mathbf{Enc}}}
\def \ODec {{\mathbf{Dec}}}
\newcommand{\SimuProc}{\mathbf{Sim}}
\newcommand{\ROProc}{\mathbf{RO}}
\newcommand{\PrimProc}{\mathbf{Prim}}
\def \stm {g}
\def \istm {\hat{g}}
\def \kts {{f}}
\def \lex {{\sf lex}}
\def \part {part}
\def \kd {{\sf{kd}}}
\def \msgdist {{d}}
\def \keydist {{r}}
\def \ind {{\sf{index}}}
\def \kprf {z}
\def \adv {{\cal A}}
\def \pwds {u}
\def \tokens {v}
\def \template{{\cal T}}
\def \vaultset{{\cal V}}
\def \ext {{\sf ext}}
\def \offset {\delta}
\def \maxweight {\epsilon}
\def \advo {{\cal A}^{*}}

\newcommand{\Chall}{\textsf{Ch}}
\newcommand{\MI}{\textnormal{MI}}
\newcommand{\MR}{\textnormal{MR}}
\newcommand{\MRCCA}{\textnormal{MR-CCA}}
\newcommand{\SAMP}{\textnormal{SAMP}}
\newcommand{\DTEgame}{\textnormal{SAMP}}
\newcommand{\KR}{\textnormal{KR}}
\newcommand{\advA}{{\cal A}}
\newcommand{\advB}{{\cal B}}
\newcommand{\advI}{{\cal I}}
\newcommand{\next}{\;;\;}
\newcommand{\TabC}{\texttt{C}}
\newcommand{\TabR}{\texttt{R}}
\newcommand{\Hash}{H}
\newcommand{\Cipher}{\pi}
\newcommand{\CipherInv}{\pi^{-1}}
\newcommand{\simu}{{\mathcal S}}
\newcommand{\prim}{P}
\newcommand{\maxguess}{\gamma}

\newcommand{\bigO}{\mathcal{O}}
\newcommand{\calG}{{\mathcal G}}

\def\sqed{{\hspace{5pt}\rule[-1pt]{3pt}{9pt}}}
\def\qedsym{\hspace{2pt}\rule[-1pt]{3pt}{9pt}}

\newcommand{\Colon}{{\::\;\;}}
\newcommand{\good}{\textsf{Good}}

\newcommand\Tvsp{\rule{0pt}{2.6ex}}
\newcommand\Bvsp{\rule[-1.2ex]{0pt}{0pt}}
\newcommand{\TabPad}{\hspace*{5pt}}
\newcommand\TabSep{@{\hspace{5pt}}|@{\hspace{5pt}}}
\newcommand\TabSepLeft{|@{\hspace{5pt}}}
\newcommand\TabSepRight{@{\hspace{5pt}}|}


\DeclareMathOperator*{\argmin}{argmin}
\newcommand{\comma}{\textnormal{,}}

\newcommand{\bits}{\{0,1\}}

\renewcommand{\paragraph}[1]{\vspace*{6pt}\noindent\textbf{#1}\;}


\newcommand{\secref}[1]{\mbox{Section~\ref{#1}}}
\newcommand{\thref}[1]{\mbox{Theorem~\ref{#1}}}
\newcommand{\defref}[1]{\mbox{Definition~\ref{#1}}}
\newcommand{\corref}[1]{\mbox{Corollary~\ref{#1}}}
\newcommand{\lemref}[1]{\mbox{Lemma~\ref{#1}}}
\newcommand{\clref}[1]{\mbox{Claim~\ref{#1}}}
\newcommand{\propref}[1]{\mbox{Proposition~\ref{#1}}}
\newcommand{\factref}[1]{\mbox{Fact~\ref{#1}}}
\newcommand{\remref}[1]{\mbox{Remark~\ref{#1}}}
\newcommand{\figref}[1]{\mbox{Figure~\ref{#1}}}
%\newcommand{\eqref}[1]{\mbox{Equation~(\ref{#1})}}
% Have to use \renewcommand because exists already in amsmath
\renewcommand{\eqref}[1]{\mbox{(\ref{#1})}}
\newcommand{\consref}[1]{\mbox{Construction~\ref{#1}}}
\newcommand{\tabref}[1]{\mbox{Table~\ref{#1}}}
\newcommand{\apref}[1]{\ifnum\camready=0 \mbox{Appendix~\ref{#1}}\else the
full version\fi}



\newcommand{\calK}{\mathcal{K}}
\newcommand{\calE}{\mathcal{E}}
\newcommand{\calD}{\mathcal{D}}
\newcommand{\calH}{\mathcal{H}}

\newcommand{\algfont}[1]{\textsf{#1}}
\newcommand{\varfont}[1]{\textrm{\small#1}}


\begin{document}
\thispagestyle{empty}

\begin{center}
{\Large SaTC: CORE: Small }\\ 
{\Large API-Centric Cryptography\\} 
\end{center}

\noindent
The prevailing mindset in the academic cryptography community is that
we produce theoretically secure and (one hopes) efficient
cryptographic primitives, and it is the job of security engineers and
software developers to turn these into real pieces of software that
are useful for applications.  Implicitly, it is their job to respect
the syntax that we formalize, and to properly implement the
provably-secure constructions that we develop.  But
the gap between what suffices as a
specification of a primitive in the literature, and an actual
implementation ---~one that is correct, efficient, and flexible enough
to support applications~--- is large.  Moreover, various 
people (e.g., Paul Kocher in his Crypto '16 invited lecture) have
voiced the concern that cryptographers do not appreciate how
difficult it is to develop good cryptographic libraries, in particular
good APIs; nor that APIs are relatively inflexible once fielded,
because changing them (say, to accomodate the latest crypto theory
results) wreaks havoc on applications that call them.

Motivated by this, we propose an alternative viewpoint: it is
\emph{our} job to respect the relatively inflexible nature of
real-world libraries and their APIs, and it is \emph{our} job to make
theoretical cryptography that is as easy as possible to implement
correctly, and that is resilent to misuse by applications. We refer to this viewpoint as
\emph{API-centric cryptography}.  This approach suggests at least
three consequences for the way we develop theory: existing formalisms
shoudl be carefully reexamined with respect to the actual APIs that
libraries export, new realizations of existing abstract primitives
should take into consideration how easy it will be to implement them
with broadly adopted libraries, and that the formalization of new
abstract primitives should include syntax that is clear about the
functionalities that will need to be implemented.  That is, syntax
that is ``API-like''.

Concretely, we identify a collection of research tasks that adhere to
this API-centric point of view.  The first group addresses the
application of the approach to secure channels, authenticated key
exchange, as well as examining standards for cryptographic APIs.  It
also includes efforts to survey existing libraries and software
artifacts, in order to understand what are the \emph{effective}
cryptographic primitives that are exported by the APIs; in particular
APIs aimed at application developers, who have relatively little
crypto expertise.  
%
The second group takes aim at developing
theoretical primitives that are misuse-forgiving.  Here too, we
propose tasks designed to better understand the landscape of misuse.
This group contains tasks to revisit so-called ``hedged'' cryptography
in a way that more closely matches practice, and to develop primtives
that are expressly meant to handle (natively) the highly structured
data that is common in practice.
%
The third group addresses an effort to produce abstract syntax for
cryptographic primitives that is more API-like.
  

\smallskip
\noindent \textbf{Intellectual Merit:} This work will develop a new
viewpoint on cryptographic primitives and protocols, one that treats the
implementation and deployment needs of cryptographic practice as a
primary guide for theory.  The work will require us to develop new styles of
abstract syntax, and new methods for specifying realizations of these
abstractions.  
%It will require new methodologies to integrate real-world insights that empirical,
%data-driven study provides, with formal security analysis, the latter
%in the vein of modern cryptography's provable security paradigm. 
We believe this work can form the foundation of a new approach to
development of secure cryptographic tools that are easier for
developers to implement correctly, and that are more robust against
misuse.  

\smallskip

\noindent {\bf Broader Impact:}  We have arranged our research program to
maximize its potential for broad impact on a number of groups. We have
experience transitioning our research results to industry, as well as to tools for the anti-censorship
activist community. We target similar impact for the proposed
work. We will actively engage industry, both to glean real-world requirements
as well as to advertise our new approaches.  This will help as well in our efforts
to use the proposed work to more broadly build bridges between the practitioner
and academic communities, through workshop development, exposure of academics to 
research questions of applied value, and educational activities stemming from
the new research methodologies we will develop. We will carefully examine
standards (IEEE and ISO, for example) and inform them of our
findings. Finally, we expect that the proposed API-centric viewpoint will find utility in a broad array
of future research efforts in cryptography.

%%
\vspace{1ex}
\noindent 
{\bf Key words:} cryptography, APIs, cryptographic formalisms,
authenticated encryption, secure channels, hedged cryptography

\end{document}
