\documentclass[11pt]{article}

%\newcommand{\bnm}{\begin{newmath}}
\newcommand{\enm}{\end{newmath}}

\newcommand{\bea}{\begin{eqnarray*}}
\newcommand{\eea}{\end{eqnarray*}}



\newcommand{\bne}{\begin{newequation}}
\newcommand{\ene}{\end{newequation}}

\newenvironment{newmath}{\begin{displaymath}%
\setlength{\abovedisplayskip}{4pt}%
\setlength{\belowdisplayskip}{4pt}%
\setlength{\abovedisplayshortskip}{6pt}%
\setlength{\belowdisplayshortskip}{6pt} }{\end{displaymath}}

\newenvironment{neweqnarrays}{\begin{eqnarray*}%
\setlength{\abovedisplayskip}{-4pt}%
\setlength{\belowdisplayskip}{-4pt}%
\setlength{\abovedisplayshortskip}{-4pt}%
\setlength{\belowdisplayshortskip}{-4pt}%
\setlength{\jot}{-0.4in} }{\end{eqnarray*}}

\newenvironment{newequation}{\begin{equation}%
\setlength{\abovedisplayskip}{4pt}%
\setlength{\belowdisplayskip}{4pt}%
\setlength{\abovedisplayshortskip}{6pt}%
\setlength{\belowdisplayshortskip}{6pt} }{\end{equation}}


\newcommand{\Adv}{\mathbf{Adv}}
\newcommand{\AdvMI}[1]{\Adv^\mathrm{mi}_{#1}}
\newcommand{\AdvHEDIST}[1]{\Adv^\mathrm{dist}_{#1}}
\newcommand{\AdvMR}[1]{\Adv^\mathrm{mr}_{#1}}
\newcommand{\AdvMRCCA}[1]{\Adv^\mathrm{mr\textnormal{-}cca}_{#1}}
\newcommand{\AdvKR}[1]{\Adv^\mathrm{kr}_{#1}}
\newcommand{\AdvSAMPRAT}[1]{\Adv^\mathrm{dte\textnormal{-}ratio}_{#1}}
\newcommand{\AdvSAMPIND}[1]{\Adv^\mathrm{dte}_{#1}}
%\newcommand{\AdvDTE}[1]{\Adv^\mathrm{dte}_{#1}}

\newcommand{\decOracle}{\textbf{Dec}}

\newcommand{\negsmidge}{{\hspace{-0.1ex}}}
\newcommand{\cdotsm}{\negsmidge\negsmidge\negsmidge\cdot\negsmidge\negsmidge\negsmidge}

\def\suchthatt{\: :\:}

\newcommand{\Prob}[1]{{\Pr\left[\,{#1}\,\right]}}
\newcommand{\probb}[2]{{\Pr}_{#1}\left[\,{#2}\,\right]}
\newcommand{\Probb}[2]{\Pr[#1]}
\newcommand{\CondProb}[2]{{\Pr}\left[\: #1\:\left|\right.\:#2\:\right]}
\newcommand{\CondProbb}[2]{\Pr[#1|#2]}
\newcommand{\ProbExp}[2]{{\Pr}\left[\: #1\:\suchthatt\:#2\:\right]}
\newcommand{\Ex}[1]{{\textnormal{E}\left[\,{#1}\,\right]}}
\newcommand{\Exx}{{\textnormal{E}}}

\newcommand{\true}{\textsf{true}}
\newcommand{\false}{\textsf{false}}




\newcommand{\getsr}{{\:{\leftarrow{\hspace*{-3pt}\raisebox{.75pt}{$\scriptscriptstyle\$$}}}\:}}
\newcommand{\getm}{{\:\leftarrow_{\mdist}\:}}
\newcommand{\getmcand}{{\:\leftarrow_{\mcanddist}\:}}
\newcommand{\getd}{{\:\leftarrow_{\ddist}\:}}
%\newcommand{\getm}{{\:{\leftarrow{\hspace*{-3pt}\raisebox{.75pt}{$\scriptscriptstyle \mdist$}}}\:}}
\newcommand{\getk}{{\:\leftarrow_{\kdist}\:}}
%\newcommand{\getk}{{\:{\leftarrow{\hspace*{-3pt}\raisebox{.75pt}{$\scriptscriptstyle \kdist$}}}\:}}
\newcommand{\getp}{{\:\leftarrow_{p}\:}}



\newcommand{\gamesfontsize}{\small}
\newcommand{\fpage}[2]{\framebox{\begin{minipage}[t]{#1\textwidth}\setstretch{1.1}\gamesfontsize  #2 \end{minipage}}}

\newcommand{\hpages}[3]{\begin{tabular}{cc}\begin{minipage}[t]{#1\textwidth} #2 \end{minipage} & \begin{minipage}[t]{#1\textwidth} #3 \end{minipage}\end{tabular}}


\newcommand{\hfpages}[3]{\hfpagess{#1}{#1}{#2}{#3}}
\newcommand{\hfpagess}[4]{
        \begin{tabular}{c@{\hspace*{.5em}}c}
        \framebox{\begin{minipage}[t]{#1\textwidth}\setstretch{1.1}\gamesfontsize #3 \end{minipage}}
        &
        \framebox{\begin{minipage}[t]{#2\textwidth}\setstretch{1.1}\gamesfontsize #4 \end{minipage}}
        \end{tabular}
    }
\newcommand{\hfpagesss}[6]{
        \begin{tabular}{c@{\hspace*{.5em}}c@{\hspace*{.5em}}c}
        \framebox{\begin{minipage}[t]{#1\textwidth}\setstretch{1.1}\gamesfontsize #4 \end{minipage}}
        &
        \framebox{\begin{minipage}[t]{#2\textwidth}\setstretch{1.1}\gamesfontsize #5 \end{minipage}}
        &
        \framebox{\begin{minipage}[t]{#3\textwidth}\setstretch{1.1}\gamesfontsize #6 \end{minipage}}
        \end{tabular}
    }
\newcommand{\hfpagessss}[8]{
        \begin{tabular}{c@{\hspace*{.5em}}c@{\hspace*{.5em}}c@{\hspace*{.5em}}c}
        \framebox{\begin{minipage}[t]{#1\textwidth}\setstretch{1.1}\gamesfontsize #5 \end{minipage}}
        &
        \framebox{\begin{minipage}[t]{#2\textwidth}\setstretch{1.1}\gamesfontsize #6 \end{minipage}}
        &
        \framebox{\begin{minipage}[t]{#3\textwidth}\setstretch{1.1}\gamesfontsize #7 \end{minipage}}
        &
        \framebox{\begin{minipage}[t]{#4\textwidth}\setstretch{1.1}\gamesfontsize #8 \end{minipage}}
        \end{tabular}
    }

\newcommand{\vecw}{\mathbf{w}}
\newcommand{\R}{\mathbb{R}}
\newcommand{\N}{\mathbb{N}}
\newcommand{\Z}{\mathbb{Z}}
\newcommand{\load}{L}
\newcommand{\coll}{\mathsf{Coll}}
\newcommand{\nocoll}{\overline{\mathsf{Coll}}}


\newcommand{\Img}{\textsf{Img}}


\newcommand{\myind}{\hspace*{1em}}
\newcommand{\thh}{^{\textit{th}}} % th
\newcommand{\concat}{\,\|\,}
\newcommand{\dotdot}{..}
\newcommand{\emptystr}{\varepsilon}


\newcommand{\round}{\textsf{round}}

\newcommand{\alphabar}{\overline{\alpha}}
\newcommand{\numbinsbar}{\overline{b}}
\newcommand{\numballs}{a}
\newcommand{\numbins}{b}


\def \calR {{\mathcal{R}}}
\def \gen {{\textsf{gen}}}
\def \rejsam {{\textsf{RejSam}}}
\def \xspace {{\cal{X}}}
\def \yspace {{\cal{Y}}}
\def \xdist {{p_x}}
\def \ydist {{p_y}}
\def \mspace {{\cal{M}}}
\def \mspacecand {{\cal{N}}}
\def \mspacebot {{\cal{M}_\bot}}
\def \sspace {{\cal{S}}}
\def \slen {{s}}
\def \kspace {{\cal{K}}}
\def \kspacesize {{m}}
\def \mspacesize {{n}}
\def \kdict {D}
\def \dictsize {d}
\newcommand{\kdist}{p_k}
\newcommand{\mdist}{p_m}
\newcommand{\mcanddist}{g}
\newcommand{\ddist}{p_d}
\newcommand{\sdist}{p_s}
%\def \kdist {{\kappa}}
%\def \mdist {{\mu}}
%\def \ddist {{\delta}}
\def \pspace {{\cal{P}}}
\def \mpspace {{\cal{MP}}}
\def \cspace {{\cal{C}}}
\def \key {K}
\def \msg {M}
\def \secret {K}
\def \seed {S}
\def \ctxt {C}
\def \ctxtpart {C_2}
\def \DTE {{\textsf{DTE}}}
\newcommand{\genprime}{{\textsf{GenPrime}}}
\newcommand{\isprime}{{\textsf{IsPrime}}}
\newcommand{\divisible}{{\textsf{IsDiv}}}
\newcommand{\LeastLesserPrime}{{\textsf{PrevPrime}}}
\newcommand{\GetPrevDiv}{{\textsf{PrevPrimeDiv}}}
\def \encode {{\textsf{encode}}}
\newcommand{\DTEis}{{\textsf{IS-DTE}}}
\newcommand{\encodeis}{{\textsf{is-encode}}}
\newcommand{\decodeis}{{\textsf{is-decode}}}
\newcommand{\DTErej}{{\textsf{REJ-DTE}}}
\newcommand{\encoderej}{{\textsf{rej-encode}}}
\newcommand{\decoderej}{{\textsf{rej-decode}}}

%\newcommand{\encodeis}{{\textsf{encode}_{\textrm{is}}}}
%\newcommand{\decodeis}{{\textsf{decode}_{\textrm{is}}}}
\newcommand{\rep}{\textsf{rep}}
\newcommand{\isErr}{\epsilon_{\textnormal{is}}}
\newcommand{\incErr}{\epsilon_{\textnormal{inc}}}
\def \decode {{\textsf{decode}}}
\def \enc {{\textsf{enc}}}
\def \dec {{\textsf{dec}}}
\def \SEscheme {{\textsf{SE}}}
\def \HEscheme {{\textsf{HE}}}
\def \CTR {{\textsf{CTR}}}
\def \encHE {{\textsf{HEnc}}}
\def \HIDE {{\textsf{HiaL}}}
\def \encHIDE {{\textsf{HEnc}}}
\def \decHIDE {{\textsf{HDec}}}
\def \decHE {{\textsf{HDec}}}
\def \encHEt {{\textsf{HEnc2}}}
\def \decHEt {{\textsf{HDec2}}}
\def \rec {{\textsf{Rec}}}
\def \ssketch  {{\textsf{SS}}}
\def \dsrec {\tilde{{\textsf{Rec}}}}
\def \dsssketch  {\tilde{{\textsf{SS}}}}
\def \dist  {{\textsf{dist}}}


\newcommand{\oddnums}{\mathbb{O}}


\newcommand{\DTErsarej}{{\textsf{RSA-REJ-DTE}}}
\newcommand{\encodeRSAREJ}{{\textsf{rsa-rej-encode}}}
\newcommand{\decodeRSAREJ}{{\textsf{rsa-rej-decode}}}
\newcommand{\DTErsainc}{{\textsf{RSA-INC-DTE}}}
\newcommand{\encodeRSAINC}{{\textsf{rsa-inc-encode}}}
\newcommand{\decodeRSAINC}{{\textsf{rsa-inc-decode}}}
\newcommand{\DTEunf}{{\textsf{UNF-DTE}}}
\newcommand{\DTEnunf}{{\textsf{NUNF-DTE}}}
\newcommand{\DTErsassl}{{\textsf{RSA-SSL-DTE}}}
\newcommand{\encodeRSASSL}{{\textsf{rsa-ssl-encode}}}
\newcommand{\decodeRSASSL}{{\textsf{rsa-ssl-decode}}}



%\def \encHE {{\sf{enc}^{HE}}}
%\def \decHE {{\sf{dec}^{HE}}}
%\def \encHEt {{\sf{enc}^{HE2}}}
%\def \decHEt {{\sf{dec}^{HE2}}}
\def \idealHE {{\mathcal{HE}}}
\def \IEnc {{\mathbf{\rho}}}
\def \IDec {{\mathbf{\rho^{-1}}}}
\def \OEnc {{\mathbf{Enc}}}
\def \ODec {{\mathbf{Dec}}}
\newcommand{\SimuProc}{\mathbf{Sim}}
\newcommand{\ROProc}{\mathbf{RO}}
\newcommand{\PrimProc}{\mathbf{Prim}}
\def \stm {g}
\def \istm {\hat{g}}
\def \kts {{f}}
\def \lex {{\sf lex}}
\def \part {part}
\def \kd {{\sf{kd}}}
\def \msgdist {{d}}
\def \keydist {{r}}
\def \ind {{\sf{index}}}
\def \kprf {z}
\def \adv {{\cal A}}
\def \pwds {u}
\def \tokens {v}
\def \template{{\cal T}}
\def \vaultset{{\cal V}}
\def \ext {{\sf ext}}
\def \offset {\delta}
\def \maxweight {\epsilon}
\def \advo {{\cal A}^{*}}

\newcommand{\Chall}{\textsf{Ch}}
\newcommand{\MI}{\textnormal{MI}}
\newcommand{\MR}{\textnormal{MR}}
\newcommand{\MRCCA}{\textnormal{MR-CCA}}
\newcommand{\SAMP}{\textnormal{SAMP}}
\newcommand{\DTEgame}{\textnormal{SAMP}}
\newcommand{\KR}{\textnormal{KR}}
\newcommand{\advA}{{\cal A}}
\newcommand{\advB}{{\cal B}}
\newcommand{\advI}{{\cal I}}
\newcommand{\next}{\;;\;}
\newcommand{\TabC}{\texttt{C}}
\newcommand{\TabR}{\texttt{R}}
\newcommand{\Hash}{H}
\newcommand{\Cipher}{\pi}
\newcommand{\CipherInv}{\pi^{-1}}
\newcommand{\simu}{{\mathcal S}}
\newcommand{\prim}{P}
\newcommand{\maxguess}{\gamma}

\newcommand{\bigO}{\mathcal{O}}
\newcommand{\calG}{{\mathcal G}}

\def\sqed{{\hspace{5pt}\rule[-1pt]{3pt}{9pt}}}
\def\qedsym{\hspace{2pt}\rule[-1pt]{3pt}{9pt}}

\newcommand{\Colon}{{\::\;\;}}
\newcommand{\good}{\textsf{Good}}

\newcommand\Tvsp{\rule{0pt}{2.6ex}}
\newcommand\Bvsp{\rule[-1.2ex]{0pt}{0pt}}
\newcommand{\TabPad}{\hspace*{5pt}}
\newcommand\TabSep{@{\hspace{5pt}}|@{\hspace{5pt}}}
\newcommand\TabSepLeft{|@{\hspace{5pt}}}
\newcommand\TabSepRight{@{\hspace{5pt}}|}


\DeclareMathOperator*{\argmin}{argmin}
\newcommand{\comma}{\textnormal{,}}

\newcommand{\bits}{\{0,1\}}

\renewcommand{\paragraph}[1]{\vspace*{6pt}\noindent\textbf{#1}\;}


\newcommand{\secref}[1]{\mbox{Section~\ref{#1}}}
\newcommand{\thref}[1]{\mbox{Theorem~\ref{#1}}}
\newcommand{\defref}[1]{\mbox{Definition~\ref{#1}}}
\newcommand{\corref}[1]{\mbox{Corollary~\ref{#1}}}
\newcommand{\lemref}[1]{\mbox{Lemma~\ref{#1}}}
\newcommand{\clref}[1]{\mbox{Claim~\ref{#1}}}
\newcommand{\propref}[1]{\mbox{Proposition~\ref{#1}}}
\newcommand{\factref}[1]{\mbox{Fact~\ref{#1}}}
\newcommand{\remref}[1]{\mbox{Remark~\ref{#1}}}
\newcommand{\figref}[1]{\mbox{Figure~\ref{#1}}}
%\newcommand{\eqref}[1]{\mbox{Equation~(\ref{#1})}}
% Have to use \renewcommand because exists already in amsmath
\renewcommand{\eqref}[1]{\mbox{(\ref{#1})}}
\newcommand{\consref}[1]{\mbox{Construction~\ref{#1}}}
\newcommand{\tabref}[1]{\mbox{Table~\ref{#1}}}
\newcommand{\apref}[1]{\ifnum\camready=0 \mbox{Appendix~\ref{#1}}\else the
full version\fi}



\newcommand{\calK}{\mathcal{K}}
\newcommand{\calE}{\mathcal{E}}
\newcommand{\calD}{\mathcal{D}}
\newcommand{\calH}{\mathcal{H}}

\newcommand{\algfont}[1]{\textsf{#1}}
\newcommand{\varfont}[1]{\textrm{\small#1}}


\begin{document}
\thispagestyle{empty}

\begin{center}
{\Large TWC: Medium: Collaborative: }\\ 
{\Large Distribution-Sensitive Cryptography} 
\end{center}
\noindent

We propose to develop a framework for distribution-sensitive cryptography
(DSC), a new viewpoint on tool design that will make connections between, and
improvements for, several previously proposed cryptographic contexts, as well as
lead to totally new techniques elsewhere. In DSC, one builds cryptographic
schemes that are context-specific to take advantage of knowledge about 
deployment environments in order to provide better security than would would be
possible with a generic tool.  Necessarily then, we will ground our development
in the exploration of a number of real-world problems for which good solutions
have historically been elusive:
\begin{itemize}
\item \emph{Brute-force attacks against password-based encryption}: Conventional
encryption schemes fall to brute-force attacks when keys are chosen poorly, as
is typical with keys derived from human-chosen passwords. We will
develop new solutions for this problem based on
%PIs recent work on
honey encryption (HE), a primitive recently introduced by the PIs. HE schemes
use estimates of expected message distributions to ensure that
brute-force attackers can't determine when they've used the right password.
%
\item \emph{Censorship of encrypted protocols}: Nation-states and others use
deep-packet inspection to identify and block conventionally encrypted
network protocols. A possible solution is steganography, where encryption seeks to have
ciphertexts distributed like ``benign'' traffic, but existing solutions are
too slow for practical use and, possibly, overkill relative to the types of
tests that can be run in settings with high traffic volume. Inspired by the
PIs' experience with format-transforming encryption (FTE), which can be viewed as
taking advantage of crude estimates of traffic distributions, 
we will experimentally characterize
adversarial abilities and a more advanced approach that we call
distribution-sensitive encryption (DSE).  DSE will strike a new
balance in the space of steganographic design, helping us achieve
security against real-world adversaries while supporting high performance. 
%
\item \emph{Securing human-generated authentication secrets}: Human-generated
secrets are noisy --- passwords are mistyped, biometric measurements are fuzzy,
etc. Conventional cryptographic primitives cannot deal with such ambiguity, and prior approaches
such as secure sketches and fuzzy extractors target security for arbitrary use
cases and, as a result, often leak too much information about secrets. 
We propose to explore a new primitive, distribution-sensitive secure sketches, 
as a way to provide better security in a number of applications.
\end{itemize}
\smallskip
In each case, we will explore development and application of a cross-cutting 
framework for guiding both construction and validation of distribution-sensitive
solutions. This will involve empirical estimates of appropriate distributions; 
conceiving appropriate formal, provable security definitions; building schemes,
and evaluating them both formally and via experimentation.  One notable aspect of
our approach will be to build robustness into our primitives.  In particular,
ensuring appropriate ``fallback'' security notions, in the case that distribution estimates
are damagingly wrong.   %further afield.

\medskip 

\noindent \textbf{Intellectual Merit:} This work will develop a new
viewpoint on cryptographic primitives and protocols, one that treats the
implementation and deployment needs of cryptographic practice as a
primary guide for theory.  The work will require us to develop new styles of
abstract syntax, and new methods for specifying realizations of these
abstractions.  It will require new methodologies to integrate real-world insights that empirical,
data-driven study provides, with formal security analysis, the latter
in the vein of modern cryptography's provable security paradigm. 
We believe this work can form the foundation of a new approach to
development of secure cryptographic tools that are easier for
developers to implement correctly, and that are more robust against
misuse.  

\medskip

\noindent {\bf Broader Impact:}  We have arranged our research program to
maximize its potential for broad impact on a number of groups. We have
experience transitioning our research results to industry, as well as to tools for the anti-censorship
activist community. We target similar impact for the proposed
work. We will actively engage industry, both to glean real-world requirements
as well as to advertise our new approaches.  This will help as well in our efforts
to use the proposed work to more broadly build bridges between the practitioner
and academic communities, through workshop development, exposure of academics to 
research questions of applied value, and educational activities stemming from
the new research methodologies we will develop. We will carefully examine
standards (IEEE and ISO, for example) and inform them of our
findings. Finally, we expect that the proposed API-centric viewpoint will find utility in a broad array
of future research efforts in cryptography.

%%
\vspace{1ex}
\noindent 
{\bf Key words:} 


