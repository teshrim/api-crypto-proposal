\begin{figure}
\centering
\fpage{0.75}{
\underline{\algfont{Decryption} API:} 
\begin{align*}
(\varfont{SentContext},\varfont{Ciphertext}) &\gets \algfont{ParseReceived}(X) \\
(\varfont{ExternalVals, InternalVals}) &\gets \algfont{Decrypt}(\varfont{ChannelContext, SentContext}, \varfont{Ciphertext})\\
\varfont{StatusMsg} &\gets \algfont{IsValid}(\varfont{InternalVals})\\
(\varfont{ChannelContextUpdates}, \varfont{ExternalVals}) &\gets
                                  \algfont{ErrorHandling}(%\varfont{ChannelContext},
                                  \varfont{StatusMsg}, \varfont{ExternalVals})
\end{align*}
} 
\caption{{\bf Expanding the traditional syntax for symmetric-key decryption $\calD
  \colon\calK \times \calH \times \bits^* \to \bits^* \cup \{\bot\}$ into an
  API-like specification.} The input~$X$ is the value received from the
  channel.  The \varfont{ChannelContext} is
  receiver-maintained information about the channel, and may contain values shared with the sender
  (e.g., the key, channel ID), along with local state.  The
  \varfont{SentContext} is context data associated to this particular
  ciphertext. The \varfont{ExternalVals} are intended to be released
  to the external caller, and the \varfont{InternalVals} are intended
  for use only within the decryption process boundary.  The
  \varfont{StatusMsg} supports decryption process logic and error
  handling.  
  The types of all input and output values are implicit.}
\label{fig:syntax-api-example}
\end{figure}
