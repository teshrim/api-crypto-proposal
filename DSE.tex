
%---------------------------------------------------------------------------------

\section{Circumventing Censorship: Distribution-Sensitive Encryption (DSE)}
\label{sec:dse} 
At a high level, honey encryption is about reproducing a target distribution
under decryption.  We now turn our attention to the matter of
reproducing a target distribution under encryption, taking as motivation the
important problem of circumventing Internet censorship. 

%\tsnote{The previous panel liked censorship circumvention as a
%  motivation.  However, they thought that FTE wasn't a viable approach to
%  thrwarting censorship, largely becauase of ``active warden''
%  attacks, so we need to frame this better.}
\heading{Background. }
For years, nation-states and other network operators have filtered
Internet traffic by inspecting IP addresses and TCP port numbers.
Citizens subjected to such filtering have employed a variety of tools
to obscure TCP/IP information---the Tor network~\cite{Dingledine04tor:the}, for example.  
However, the introduction of deep-packet inspection (DPI)
technology has enabled more nuanced censorship.  By peering into
packet payloads, censors may now filter traffic based on what
higher-level protocols and application data are being carried.
Governments have used DPI to block based on blacklisted keywords
appearing in HTTP request headers~\cite{ConceptDoppler}, TLS traffic~\cite{TLS:iran_block}, 
Tor traffic~\cite{Tor:iran_block-2011,Tor:iran_block,Tor:china_block_one,Tor:china_block,Tor:china_active_probe,Winter2012,Clayton06ignoringthe}, and 
Skype VoIP~\cite{China_skype_ban,UAE_skype_ban,DAC1142,Santolalla:2007,Dorfinger2010,Domingo2011}. 

An obvious response is to apply steganographic tools to packet payloads. 
However, traditional tools have been ad hoc and 
offer uncertain security (at best).  Those with well reasoned security
guarantees~\cite{Hopper:Provable_Stego} provide too little bandwidth for many real-world
applications, like surfing the web.

An alternative is to simply obfuscate packet payloads by using conventional
encryption on them; after all, ciphertexts should look like random
strings.  Recent work, however, shows that encrypted (or
compressed) payloads are easily detected~\cite{KBMP13}.
In any case, payloads randomized by conventional encryption will not pass
whitelist-based filtering, which allow only payloads that conform to
particular protocol formats.  


%\heading{Efforts to circumvent DPI-powered Censors. }
\heading{Protocol mimicry. }
Recently, a sheaf of more sophisticated obfuscation techniques have
been proposed, largely in support of the Tor Project's
pluggable transports architecture~\cite{Tor:pluggable}. This provides 
a layer below existing Tor encryption for mechanisms aimed at obfuscating 
the existence of Tor traffic.  Several of these techniques seek to
produce traffic that mimics particular applications or protocols. 
Among these, Skypemorph~\cite{Skypemorph} reformats 
traffic to look like Skype VOIP traffic. Stegotorus~\cite{Stegotorus}
encrypts and (more or less) directly embeds data into fixed
templates of HTTP message headers, Javascript objects and PDF files,
{a la} traditional steganographic techniques.  Format-transforming
encryption~\cite{Dyer-2013}, which we will discuss shortly, induces
a (secretly keyed) mapping from plaintexts to elements of user-specified sets,
e.g., strings
in the language of a regular expression capturing the structure of 
HTTP-request messages.


%\subsection{Addressing mimicry (parroting) challenges}
%\heading{???}
In principle, censorship-circumvention
systems that rely heavily on protocol mimicry are fragile in the
presence of sufficiently sophisticated censors.  Recent
work~\cite{HoumansadrBS13} observes that, in the limit, such a system
must be able to mimic every aspect of a protocol, even a specific
implementation of the protocol, including its quirks and bugs.  And
the mimicry needs to be sustained against an ``active warden,'' a
censor that performs active and adaptive probing of communicating
parties.

While sympathetic to warnings raised by this observation, we believe
that protocol-mimicry can be of significant use {\em in
  practice}, even if the mimicry is far from total.  This belief is
supported by three observations.  First, there is little evidence that real
censorship adversaries do, or even can, carry out highly sophisticated
mimicry-catching tests at line speed.  (Work on implementing the
well-known Bro system~\cite{Bro} at the university level might 
suggest the contrary~\cite{DFPS-bro}.)
Second, it is not known what
fraction of real traffic would be flagged as ``mimicry'' by such
tests.  Finally, recent work~\cite{Dyer-2013} shows that real DPI devices
\textit{are} vulnerable to so-called protocol-misclassification attacks. 

\heading{Format-transforming encryption.} In support of the last observation, PIs
Ristenpart and Shrimpton recently introduced format-transforming encryption
(FTE)~\cite{Dyer-2013}, which we can now view as a kind of DSC primitive that is
well suited for the payload obfuscation task.  This is because FTE enables
precise control of the format of ciphertexts, i.e.\ how they will appear to an
observer on the wire.  

The FTE system built in~\cite{Dyer-2013} used the fact that modern
DPI systems seem to rely heavily on regular expression matching, or ad hoc tests
that are reasonably approximated by regular-expression matching, in
carrying out traffic classification. In particular, our FTE system
allowed users to specify the desired format of the (ciphertext) packet
payloads by inputting a regular expression. Subsequently, the FTE
ciphertexts were guaranteed to be random
strings from the language described by that regular expression.\footnote{Follow up work~\cite{luchaup2014formatted} gave new algorithmic techniques that
admitted more efficient regular-expression-based FTE, as well as
expanding the scope of formats to cover context-free languages.} 
 

%In some ways, the FTE formalism is the natural result of elevating
%the goal of DPI-avoidance to a first-class security property of traditional
%encryption.
%\tsnote{Needs updating to reflect CCS paper.  Drive home the point
%  that we've proved FTE viable in lab tests against real DPI; mention
%  the month-long China test.}
In~\cite{Dyer-2013}, we showed that a battery of six modern DPI systems
(including an commercial, enterprise-grade hardware DPI system) can be
caused to classify FTE ciphertext messages as any protocol that we
chose, no matter what was the actual underlying plaintext.  This was
done by providing FTE with appropriate regular expressions.  In the
simplest cases, these were lifted directly from the DPI system, or
created manually using knowledge of protocol RFCs.  In other cases,
the regular expressions were learned, by a simple procedure, from
network traces.

Moreover, we were able to set up an FTE-powered client inside of
China, and tunnel arbitrary plaintext traffic (including Tor, and HTTP
traffic from banned sites) to and from an FTE-powered proxy in the
USA.\footnote{The client ran tests every five minutes for one month,
  with no sign of detection by the GFC.  After
  one month, we shut it down.} 

We note that FTE is one of the few proposed obfuscation techniques
actually integrated into the Tor distribution as a pluggable
transport, and the only one that targets both blacklisting
\textit{and} whitelisting filtering rules.\footnote{The other fielded pluggable transports rely on payload randomization.}
%Neither Skypemorph nor StegoTorus are fielded by Tor.



\subsection{From FTE to DSE}
\label{subsec:beyond_FTE}
The initial successes of FTE are encouraging because they show that
regular-expression-based DPI can be fooled, and by systems that also
support typical Internet usage (ie., watching YouTube videos).
Moreover, our investigations suggest that regular-expression-based DPI
is the norm in practice, as do some manufacturer
documents, e.g.~\cite{bluecoat-best-practices}.

That said, FTE targets only the goal of
payload obfuscation, and achieves this producing ciphertexts that mimic \textit{individual} 
protocol messages.  It makes no effort to respect properties that
would be expected over an ensemble of real protocol messages, i.e.\
distributional issues.  Fundamentally, this is because FTE ciphertexts
are \textit{uniform and independent} samples from some language of strings.
Thus, a \textit{sequence} of FTE ciphertexts is unlikely to display,
say, a distribution of message lengths that mimics reality.  This shortcoming may foster
detection by traffic analysis attacks.  Moreover, and in contrast to conventional
encryption, formatted ciphertexts have observable structure by design.
As a result, DPI may come to support more sophisticated statistical attacks
based on protocol message structure or content.

\heading{Modeling real adversaries. }
We do not yet know if these kinds of attacks are a real
threat, now or in the near future, because the community lacks
understanding of real-world DPI tools and their capabilities at scales
of interest.

We do know that operating at wire speed places significant constraints on what
computational tasks can be carried out.  Consequently, nation-state
censors seem to perform two-stage processing.  The first stage
operates at wire speed, where DPI is used to select out ``suspect''
traffic for further, more computationally intensive analysis in the
second stage.  As we've already said, there is some evidence to
suggest that regular-expression matching
forms the core of wire-speed processing at nation-state traffic
volumes.  But we do not in fact know how accurate this is for
current DPI, nor how predictive of the future.  (For example, if
regular-expression-based tests are the current norm, is it a small
step, or a large jump, to support tests requiring full payload
parsing?) This brings us to our next task:
%  
\begin{task} 
\label{task:dpi-study}
We will explore the performance, capabilities and
  potential configurations of DPI products (e.g., Blue Coat ProxySG
  and Packetshaper) known to be used in censorship. 
\end{task}
\noindent
Documentation for commercial DPI tools is typically not
directly disclosed by manufacturers.  
\iffalse{
But documents have in fact leaked via,
e.g., Wikileaks.\footnote{See, e.g.,
  http://wikileaks.org/spyfiles/files/0/226\_BLUECOAT-SGOS\_CMG\_4.1.4.pdf.}
offer one way in.
\fi
PIs Ristenpart and Shrimpton, however, along with other
members of the circumvention research community and a major industrial
partner are also part of a
burgeoning effort to establish a DPI lab allowing direct experimental access to a
number of systems. To see where DPI systems may be going, we 
will also look to research efforts in this space, and seek collaborations where
appropriate with colleagues at our respective institutions with relevant
expertise.
%\tsnote{More about how we plan to do this?  This seems kind of weak, at the moment.} 

We will also carry out a measurement study, using traffic from our three
respective institutions as well as external sources, to empirically characterize
what ``benign'' traffic looks like, along various dimensions.  (PI Ristenpart
already has a full-packet-capture network tap in place at the University of
Wisconsin. All use of it is guided by appropriate IRB exemptions or reviews.)
This corpus will also support investigations of efficacy of distinguishing
attacks on FTE, DSE and other obfuscation methods. All this will require some
engineering effort to develop software analysis tools for simulating attacks.
The resulting software framework will be made available for other researchers to
use.

\ignore{
In addition to better understanding how these devices carry out
classification, we will work to understand their classification
precision.

the tool may still be useful in settings where a large false-positive rate
cannot be tolerated, say, due to political or economic ramifications.  Of
course, what constitutes a large fraction is subjective.  But without empirical
security analyses, one cannot make even crude judgments about the potential
collateral damage caused by the test. 
Similarly, attackers in this setting are often resource-constrained, and
so we will assess experimentally the whether theoretical attacks work
as well in practice.
}

These empirical investigations will help to shape formal models of DPI-powered
adversaries.  In turn, these formal models will enable principled security
arguments and, in some cases, formal proofs about future designs. Our starting
point is prior work by Hopper, Langford and von
Ahn~\cite{Hopper:Provable_Stego}. They provided the first formal study of
steganographic systems, including giving semantic-security style notions of
security.
%Unfortunately, their notions come with some strong negative results,
%essentially that no scheme can meet the security goal without being able to
%sample the true target distribution perfectly.  In practice, this is likely to
%be unpalatable, if not completely unrealistic.
Their security notion is very stringent, assuming that the adversary (in this
case, the DPI-powered censor) has complete knowledge of the distribution of
traffic on the channel at all points in time.  While conservative, this seems
unfounded in practice, especially at nation-state volumes of traffic.  What's
more, because the adversary is so powerful in their notion, they are able to
prove security only for inefficient schemes.
%Tunnelling plaintexts through existing implementations of applications or
%protocols \textit{may} be able to approximate this, although there are
%significant development challenges to be overcome in this approach.  Almost
%certainly, the resulting system will be tightly bound to the existing
%implementation, making it inflexible.  Moreover, if the tunneling application
%or protocol is banned (for example, if one was tunneling through Skype when it
%was banned in [where? when? citation?]), then the entire system must be
%rethought.  (We will speak to ``tunneling'' approaches, which would seem to get
%closest to this, at the end of the section.)
Establishing models and security notions that more accurately reflect the
realities of real DPI-powered censors should have the side-effect of allowing us
to prove security for more efficient schemes.

\begin{task} \label{task:formal-attack-models}
%(Formally modeling resource-bounded classifiers.) 
Drawing on complexity-theoretic characterizations of language recognition (e.g.,
circuit complexity), we will seek to model formally and concretely the space of
traffic classifiers (steganography detectors) implementable by a space- and
computation-bounded censor. We will aim to construct DSE schemes that shape
traffic that is provably undetectable within the resulting models.
\end{task}


\heading{Generalized DTEs and DSE.} 
We will also set ourselves to building the next-generation of steganography
primitives that are fast, yet secure relative to real-world adversaries. 
Towards this end, we will start by generalizing a primitive from the study of honey
encryption, namely DTEs.
Consider two sets $\xspace,\yspace$ equipped with distributions $\xdist,\ydist$
respectively. A \textit{generalized DTE} would be tasked to encode
$\xdist$-distributed samples of~$\xspace$ so that they appear to be
$\ydist$-distributed samples of~$\yspace$, and vice versa.
Setting $\yspace$ to bitstrings and~$\ydist$ to uniform, and~$\xspace$ 
to prime numbers and $\xdist$ to uniform gives the DTE we proposed
previously~\cite{HoneyEnc-EC:2014}. 

This generality now allows us to recognize the primary FTE constructions as an
instance of what we call \emph{encrypt-then-DTE}.  Setting $\xspace$ to
bitstrings and~$\xdist$ to uniform, and~$\yspace$ to some target language with
distribution~$\ydist$, one could use encrypt-then-(generalized-)DTE to build FTE
schemes that turn plaintext into ciphertexts that are samples from~$\yspace$
that appear to be~$\ydist$-distributed.  Indeed current FTE is encrypt-then-DTE
with a DTE whose outputs are uniform samples from a specified regular or
context-free language. Being able to treat non-uniform
distributions, would be a significant step forward: it would specifically help
to defuse statistical attacks by DPI adversaries that look for uniform
distributions.  We call the resulting approach distribution-sensitive encryption
(DSE), which we view as an efficient kind of steganography that need not fool
human observers. 

%current state of the art in FTE, where $\yspace$ is a specified regular or
%context-free languge, and $\ydist$ is uniform.

%Towards this, we note that FTE can be viewed as an instantiation of \textit{encrypt-then-DTE}:
%the plaintext is first encrypted using a conventional
%aencryption scheme, and then a specific, bijective
%encoder efficiently maps the (uniform) ciphertext to elements of the
%language.
%\tsnote{I haven't said specifically how this happens, ie.\
%  rank/unrank. I don't see a strong reason to go to this level of
%  detail.} 
%Thus, even if the strings in the language
%nicely resemble (say) real HTTP-request messages, FTE will return a
%ciphertext that is a uniform sample of these strings.
%To prepare for future challenges, we desire to control not only the
%elements of the language (ie., the ciphertext format) but also the
%\textit{distribution} by which these elements are sampled. 

The encrypt-then-DTE viewpoint on DSE proves to actually capture prior work on
steganography as well: Cachin proposed what amounts to an encrypt-then-DTE
construction, where the DTE was based on universal coding schemes, to achieve
information-theoretically secure steganography~\cite{Cachin:Info_Theory_Stego}.
A similar approach was taken in follow-up work by Backes and Cachin for
public-key steganography~\cite{BC04}.
As far as we are aware, this work has not yet led to any practical mechanism.

To achieve generalized DTEs, one might hope for schemes that work for 
arbitrary sets $\xspace,\yspace$ with respective
distributions $\xdist,\ydist$.
Unfortunately, it is not hard to see that not all pairs of sets/distributions
can be supported.
Correctness mandates the ability to invert, so building a DTE that maps from
a high-entropy distribution to a very low-entropy one will, in general, be
information-theoretically infeasible. 
Generalized DTEs will provide a principled way to explore 
bandwidth/security trade-offs and, ultimately, lead us to understand the foundations of these
trade-offs. 

\begin{task} 
\label{task:foundations-gen-dte}
%Informed by our broader exploration of what generalized
%  DTE schemes can achieve (Task~\ref{task:gen-dte}), we will consider
We will formalize generalized distribution-transforming encoders
(DTEs) and appropriate security notions. 
We will investigate fundamental trade-offs between efficiency and security,
providing formal bounds showing the impossibility of certain
combinations of input/output distribution pairs. 
\end{task}

\noindent With these boundaries in place to help guide us, 
we will turn to building generalized DTEs, particularly ones  
suitable for the anti-censorship setting.  Guiding this search will be the
the insight that in many cases an imperfect approximation of the covertext
distribution will be sufficient, as per our adversarial modeling discussed
above.
 %The result will be
%that encodes uniform bit strings to non-uniform outputs can be constructed using
%rejection sampling: simply run $\gen(R\concat M)$ 
%We will investigate
%possibility and impossibility results.

\begin{task}
\label{task:use-gen-dte}  We will explore constructions of generalized
DTEs for practical steganographic encryption.  %finding positive examples when possible and formally ruling out
%efficient constructions elsewhere. 
We will explore 
  what can be done \textit{efficiently} in the censorship
  circumvention setting, in particular relative to realistic attack models (Tasks~\ref{task:dpi-study},\ref{task:formal-attack-models}).
\end{task}

%We note that this moves us a step closer to
%full-blown provably secure stegonagraphy in the sense of Hopper, Langford, and
%van Ahn~\cite{Hopper:Provable_Stego}.

\noindent An oft-suggested mechanism for protocol steganography is tunneling
(c.f.,~\cite{HoumansadrBS13}).  That is, given a full implementation of some
cover protocol, one can submit encrypted data as protocol messages to the implementation. This would seem to provide very high-fidelity mimicry, at least
for one implementation and when the tunneling protocol is lossless. 
Even so, tunneling may still admit detectable
discrepancies with regards to normal use of the implementation. 
For example, if the implementation applies
lossless compression to incoming messages, then tunneling in this fashion will lead to
larger-than-normal protocol transcripts. 

\begin{task}
\label{task:tun-dte}  We will explore the use of tunneling as a DTE, comparing
its efficiency and security to our de novo DTE constructions.
%finding positive examples when possible and formally ruling out
%efficient constructions elsewhere. 
\end{task}

\heading{Mimicry above the message-level.}
Finally, we note that we have been discussing statistical attacks on
FTE at the level of protocol messages. Censors with
resources sufficient to maintain a greater amount of state could mount
statistical attacks at the connection level, e.g.,
checking that a sequence of upstream and downstream FTE ciphertext
protocol messages observe expected semantic patterns.  We do
not yet understand how much state would be required to mount
connection-level analysis at nation-state traffic volumes, although we will
explore this in our evaluation of real-world DPI adversaries (see
Task~\ref{task:dpi-study}).  We will use what we learn to 
push distribution matching behaviors ``up the stack'' in a way that retains
efficiency.

\begin{task}
\label{task:stateful-dte}
We will explore stateful DSE, where the distributions to be matched
are a function of state including previously exchanged messages---both the mimicked messages and the actual plaintexts.   
\end{task}
\noindent
A natural way to approach this is to build sender and receiver state
machines that determine which distribution will be
used for the next message. Task-structured probabilistic I/O automata~\cite{cgUCL-CCKLLPS06}
(or more general probabilistic
automata~\cite{Stoelinga02anintroduction}) 
may provide a convenient
formal model for describing these stateful behaviors. 

Stateful DSE can also take advantage of published state-machine models of various
protocols.  They also allow for things such as mimicking the number
(and type) of connections over which to send protocol messages.

%\heading{Active adversaries.}
%\tsnote{Insert text about dealing with active adversaries, if we want
%  to address this at all.}
Finally, we note that working above the message level also provides
opportunities for dealing with active adversaries, which may send
probing messages to possible proxy servers in order to see
how they respond.  The Great Firewall of China is known to employ
such tactics to help find and blacklist Tor servers~\cite{Winter2012}. The
automata can include transitions to error states that result in innocuous 
behavior  when malicious messages 
are detected (e.g., by a message authentication failure when the client and
server share a key).

\ignore{
Beyond simply blocking traffic, the Great Firewall of China actively
engages with suspected Tor network edge-nodes.  It 
inspects packet payloads for a particular 48-byte string that is uniquely
associated with a recent version of Tor.
Upon finding a packet with this string, it forces both ends to drop
the TCP connection, and subsequently attempts to carry out a Tor
handshake with the destination IP address.  If it determines that
the destination indeed `speaks' Tor, the GFC adds that IP address to
an blacklist (facilitating easier blocking in the future)~\cite{Winter2012}.
%Skype can likewise be easily detected using 
%known fingerprints~\cite{DAC1142,Santolalla:2007,Dorfinger2010,Domingo2011}. 
%Similar attacks can be used against other possible circumvention tools, such as IPsec
%tunnels. 
}

\ignore{
\heading{Previous, provable-security work on Steganography.}
Our DSE formalization shares commonalities with previous primitives targeting
provably-secure steganographic systems~\cite{Hopper:Provable_Stego}.  But ours
cleaves closer to reality than prior works because, for example, it mandates
that DSE schemes must rely upon a compact representation of the true target
distribution.  In practice, a real scheme would carry out some offline analysis
of benign traffic in order to later mimic it online.  Also, our DSE
formalization will be a generalization of FTE, so we will be able to leverage
results and techniques established there.

We also propose to establish new notions of security for DSE schemes. 

We also propose to build DSE schemes, and to establish their provable-security
guarantees with respect to these security notions.  We will give some initial
ideas in this proposal how we might construct DSE schemes; one is a method to
generically convert a traditional encryption scheme into a DSE scheme via rejection
sampling mechanisms, the other is a ``dedicated'' DSE construction, based on Markov Chain Monte Carlo
techniques.  In both cases, we propose to characterize what kinds of target
distributions can(not) be accurately approximated, and likewise if we demand
fast DSE.  The idea of using MCMC tools, and the like, to build cryptography is
particularly intriguing.
}


\iffalse
\heading{Protocol tunneling.} 
Primitives like FTE can be seen as providing lightweight (and
controllable) steganography on top of encryption, in order to achieve
some protocol mimicry at the message-level.  DSE, and then stateful DSE,
can be viewed as targeting increasing levels of fidelity in mimicry.
At the extreme end of this range is tunneling, which is to
embed messages into the user inputs of an existing implementation of a
protocol or application, essentially using the latter as a black box.
While excelling at realism, this approach is not without its own challenges. 
Almost certainly, the resulting system will be tightly bound to the
existing implementation, making it inflexible to changes in its
operating environment.  (For example, if one was tunneling through Skype when it was
banned in China~\cite{China_skype_ban} or the UAE~\cite{UAE_skype_ban}, then the entire system must be
rethought.)  

Nonetheless, tunneling may offer a viable approach, and
the techniques we explore in this proposal (e.g.\ DTEs)
can support tunneling schemes.  One example challenge 
faced by tunneling methods is proper shaping of input messages, so that the tunnel
traffic does not leak information about what is being tunneled. 
Our work can offer a principled approach to message shaping. 
%and a metric by which to measure
%effectiveness against suitable model traffic, e.g., videoconference
%samples obtained in the wild. 

\tnote{Do we want a task?}

From FTE to tunneling, we will investigate a variety of design points along
the protocol-mimicry continuum, with a focus on exposing tradeoffs
among security, performance, ease-of-implementation and operational flexibility
\fi

%Recent, early investigation of this approach, e.g.,~\cite{LiSchliepHopper:2014}, 
%has adopted an ad hoc approach to shaping message distributions. (E.g., the Facet system in~\cite{LiSchliepHopper:2014} modifies YouTube videos; the effectiveness of the modification algorithm is tested against a basic statistical classifier.) The techniques we explore in this proposal, such as Distribution Transforming Encoders (DTEs) (see Section~\ref{}) offer a principled approach to message shaping and a metric by which to measure effectiveness against suitable model traffic, e.g., videoconference samples obtained in the wild. 



%%%%%% GRAVEYARD %%%%%%%

\ignore{
\subsection{Our FTE Research and Deployments}
\tsnote{We take FTE as a starting point, given demonstrated
  successes.  But there's still a lot to do...}
By intertwining the development of new theory with implementation and experimental 
evaluation, we propose to construct empirically-driven, provable-security foundations
for the DPI-avoidance problem.  New DPI-avoidance mechanisms will be both a component of 
this construction process, and a result of it.  We will implement these mechanisms and 
carry out experimental analysis of their efficacy using real network data.
including:
\begin{newenum} 
\item Algorithmic improvements~---~our initial FTE scheme
employs some classic results of Goldberg and Sipser~\cite{GS85}, but uses a
time-space tradeoff in order to reduce the on-line complexity of encryption and
decryption.  However, we found that the amount of space required to hold
precomputed tables would be too large for widespread use.  Thus we propose to
research new time-space tradeoffs.
\tsnote{Algorithmic improvements for key components of FPE and FTE
  schemes.  Talk up the ``newrank'' project.} 

%
% \tsnote{I don't know if we want to keep the
%   CFG/template grammar stuff.  We did use it a little in the CCS'13
%   paper, but I don't think our understanding has advanced much since
%   the aborted USENIX'13 submission.} 
% \item Increased format control~---~our FTE scheme uses regular expressions to
% compactly describe the set of ciphertext formats we allow.  Qualitatively,
% these initial results were positive, showing that we could transfer data
% through an HTTP whitelisting tool $\lsevenfilter$\cite{l7-filter}.  This
% supported our belief that FTE can provide useful stegonagraphic properties and
% good bandwidth.  Still, this DPI is quite simple, and more picky DPI techniques
% would likely have detected our ``HTTP'' traffic as being generated by FTE.
% What is needed, both for better fidelity and increased bandwidth, is a more
% descriptive representation of the set of ciphertext formats.  To that end, we
% propose to investigate CFG-based representations; in particular, what we call
% \textit{template grammars}. 
%

\item Increased format control~--~ [[Talk about developing higher-level
  logic to control the current format, the length of ciphertext
  messages, etc.]]

% \tsnote{We got dinged in the panel reviews on this next point.  First,
%   they didn't think it was realistic for people to learn from local
%   traffic.  Second, they thought us not prepared to do any serious ML
%   work, even suggesting to add an ML person to the proposal.}
% \item Learning ciphertext formats from real data~---~our regular expression
% representations of ciphertext formats were initially hand-crafted, based on
% studying real data and protocol specifications.  Eventually, we developed a
% rudimentary method for building regular expressions from real data, and then
% hand-tweaking the expressions afterwards as necessary.  But really what one
% wants is a well specified algorithm for learning ciphertext formats from a
% large, representative corpus of collected data.  We propose to develop
% algorithms that, given some initial structure to build upon, can learn full
% CFG-based representations.
%
\item Provable-security analysis~---~our initial empirical analysis of FTE has
been very useful, and now we would like to develop formal notions of security for
FTE schemes, and establish their provable-security guarantees.  We propose to
do this.
%
\end{newenum}


\ignore{
\begin{newenum}
%
\item \emph{How effective are traffic analysis attacks in the censorship
setting?} Prior work~\cite{Skypemorph,Stegotorus} has analyzed efficacy of
traffic analysis (TA) attacks in which the number of packets, their length,
and other information not obfuscated by traditional encryption is exploited
by the attacker. But no analysis has been conducted of whether such attacks 
are efficient enough or have low enough false positive rates to be deployable by 
network operators.
For example, prior work by the PIs gave fast TA attacks that achieve 
accuracy of around 80\% in identifying websites --- but even this 
may lead to an unacceptable false positive rate by network operators. 
%

\item \emph{Do more direct attacks exist?} Even should traffic analysis be 
undeployable, there may be more direct routes to attack. Our preliminary
work, for example, surfaces a simple attack that can 
identify Stegotorus traffic with a true positive rate of \fixme{92\%} and
a false positive rate of less than \fixme{0.01\%}. 
%
\item \emph{Can we provide formal foundations for DPI avoidance?} 
The above points to the need for new, formally sound approaches to building 
defenses against DPI-based protocol identification. 
\end{newenum}

%Our preliminary work surfaces attacks against
%some, and for others the existing security analyses, when present,  are
%unsatisfying.\authfnote{TomR}{We should check this--- skypemorph seems the
%outliar here?} Part of the complexity in evaluating obfuscation mechanisms,
%including our preliminary attacks,  stems from the fact that the research
%community does not have a clear understanding of what attacks existing and
%future DPI systems can actually deploy.
}

}
