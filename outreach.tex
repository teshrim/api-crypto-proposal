\section{Curriculum Development Activities, Outreach, and Result Dissemination}
\label{sec:outreach}
\ignore{
Our proposed research will be conducted in order to maximize  broader impact on
education, student development, and the world.  This will be facilitated by
active engagement with the censorship circumvention activist community, with
industry practitioners working in technology spaces that might benefit from DSC
tools, and community building for academic applied cryptography.
}


\paragraph{Curriculum development.}  
A key aspect of our work will be in improving curriculum to ensure students
have the right skill sets to tackle real-world problems using techniques from applied cryptography. 

PI Juels is developing a new masters-level security course at the new Cornell
Tech campus over the course of Fall 2014; given the industry-focused mission of
the campus, this course aims to impart security principles and knowledge,
including applied crypto, specifically to budding technology entrepreneurs. He
is also mentoring students in the campus's Company Projects program, guiding
them in security problems brought to the campus by
industry partners.

PI Ristenpart will integrate DSC-style viewpoints into his teaching at
University of Wisconsin. For both a PhD level and a senior undergraduate
level, he will prepare appropriate lectures covering topics pertaining to the
research conducted in this grant. The particular educational goal will be to
encourage students to learn how to think, in a principled fashion, about
contexts when performing security mechanism design. 
%Fall 2013. It will focus on preparing students to work at the cutting edge of
%research in security, and will have a particular emphasis on techniques cutting
%across the traditional boundaries of theoretical cryptography and systems
%security.  He has already had success developing an undergraduate security
%course that after just two years is already one of the most popular in the
%department, and equips students with the perspective needed to find and fix
%security problems.  
%Beyond development of undergraduate security curriculum, he has in the past
%developed a graduate course on applied cryptography, with an emphasis on
%theoretical techniques of value to practical cryptography, and a graduate course
%on broader topics in computer security incuding the interplay between
%cryptographic theory and practice.

%During the period of PI Shrimpton's first NSF grant, he developed a graduate
%course in modern cryptography at Portland State.  Motivated by problems he had
%encountered as part of his research, and frustrations he faced teaching
%cryptography,  he developed ``Counting, Probability and Computing'' during the
%period of his CAREER award.  This time, he 
PI Shrimpton will develop a course in cryptography
suitable for undergraduates.  This course will focus less on proofs and
formalisms than his graduate course, and more on applications and developing a good sense of
cryptographic hygiene.  Development of this course is partially
motivated by local industry needs.
%A major motivation for this new undergraduate course are
%the continued complaints heard from local employers, especially Intel, about new
%hires lacking any kind of security-awareness.  
In fact, Shrimpton will work with Dr.\ Jesse Walker of Intel to develop this new course.

\paragraph{Outreach and diversity.} We believe that scientific
communities are most productive when they include researchers from a wide
variety of backgrounds: science disproportionately benefits from a diversity of
viewpoints.  Towards this end,  we will make an effort when attracting students
to especially target women and underrepresented minorities.  The PIs have
already had success in this regard.  Shrimpton has mentored Mrs.\ Tashell
Kelly (undergraduate), 
Ms.\ Morgan Miller (MS 2010), Ms.\ Erin Chapman (MS 2012); 
%Alex Ross (MS 2009, now at Google);  is currently advising
%Mr.\ Kevin Dyer (PhD, expected 2014) and Robert Terashima (PhD, expected 2014); 
and served as a committee member for Mrs.\ Nichole Schimanski (PhD, 2014).
Ristenpart advised Ms.\ Alexis Fisher (MS 2013)
and Ms.\ Chih-Ching Chen (PhD). % Mr.\ Venkatanathan Varadarajan (PhD, expected 2015); 
%Mr.\ Adam Everspaugh (PhD, expected 2017); Mr.\ Robert Jellinek (MS, expected 2014); 
%and has already graduated several masters students 
%Mr.\ WesLee Frisby (MS 2012, now at Sandia National Laboratories), Mr.\ Thawan
%Kooburat (MS 2012, now at Facebook), Mr.\ Benjamin Farley (MS 2012, now at Amazon).
Both PIs are advising or already graduated several other students, now working
at Google, Tektronics, Facebook, Sandia National Laboratories, Amazon. 

%The profile of the students who study computer security at
%Portland State does not represent the diversity of the surrounding community.
%To address these issues 
PI Shrimpton is working with local academic and industry colleagues 
to reach potential CS students earlier in the educational pipeline.  Portland
State recently hosted a cybersecurity summer camp, part of an program
sponsored by the Department of Homeland Security, which is scheduled
to run for (at least) the next two years.  Shrimpton and colleagues
are currently developing plans for an outreach center that fosters 
relationships with local K-12 educators. 


\paragraph{Interaction with industry.} Our research, if successful, will improve
security in a number of critical contexts. To facilitate that, we maintain an
active network of industry contacts, which, among other things, helps us see and
navigate around potential deployment roadblocks.  Through his industry
connections as former Chief Scientist of RSA, The Security Division of EMC, PI
Juels is helping build ties more broadly between the scientific community and
those interested in commercial deployment; he has ongoing research
collaborations and/or deployment projects, for instance, with colleagues at
Cisco, Qualcomm, Microstrategy, and Two Sigma, mainly in applied cryptography.

\paragraph{Technological impact and software dissemination.}
This project will have its success measured in large part by
the degree to which the research impacts real-world artifacts.  
The PIs have a track record in this respect: FTE now ships as
part of the Tor browser bundle and is integrated with Google Idea's
uProxy product. PI Shrimpton's work on tweakable ciphers is now part of
products sold by Voltage Security. PI Ristenpart's work on format-preserving
encryption formed the basis for algorithms used widely in industry to protect
credit card numbers and other sensitive data. His work on privacy-preserving
device-tracking led to open-source software downloaded over 100,000 times. 
His work finding new types of vulnerabilities in cloud computing and elsewhere
has led to security fixes in a variety of products. PI Juels ran the research program at RSA and was responsible for the translation of a range of cryptographic research innovations into products.
%Eric Schmidt, executive chairmain of Google, recently gifted nearly
%\$100K to Portland State in recognition of the promise of FTE.


%In particular, we intend to implement our new countermeasures
%in real systems. See the attached supplement on Transitions, which describes
%our proposed work on building systems based on the FTE and DME technologies. 
%Our goal is that, several years from now, people across the world
%will be better able to evade censorship due in part to the proposed work. 
%The success of this project should certainly be measured to a significant degree in
%terms of its impact on real-world secure systems. The
%development of practical and secure RNG systems for 
%both virtualized and traditional environments will be of
%great utility in building secure systems. 

To facilitate technology transfer and impact, 
we will make public and open-source the software prototypes that result from the
proposed research. We are strong believers in the idea that making available software 
produced in the course of publicly-funded research accelerates scientific
advancement, technology transfer, and education.
We have a track record of doing this as well: 
%record of releasing publicly software prototypes for our research, including
%NSF-funded research.  For example, 
our software implementations for FTE are open
source (see \url{http://fteproxy.org} and \url{https://libfte.org}). 
%We will
%continue in this tradition with all the software developed in the course of 
%this proposal.
%ongoing work on HE implementations, such as SweetPass, will be released publicly
%as well.  We will advertise these software packages to our industry contacts.
%We note that PI Ristenpart has experience successfully releasing software tools~\cite{RMKK08}
%that have been used by more than 100,000 people.

%We will disseminate many of our research results as software prototypes, as
%the PI's have successfully done in the past~\cite{swift06}.




%We also have a strong commitment to open-source software, and have a track
%record of releasing publicly software prototypes for our research, including
%NSF-funded research.  For example, our software implementations for FTE are open
%source (see \url{http://fteproxy.org} and \url{https://libfte.org}).  Our
%ongoing work on HE implementations, such as SweetPass, will be released publicly
%as well.  We will advertise these software packages to our industry contacts.





\paragraph{Developing the scientific community.} An important part of our longer
term work is development of the applied cryptography research community, which
requires integrating better disparate disciplines within computer science. We
particularly target expanding the interaction between those building and
deploying systems and the cryptographic theory research communities.  PIs
Ristenpart and Shrimpton are both on the steering committee of the Real World
Cryptography workshop, now in its fourth year; we intend to continue
over the lifetime of this grant.  This workshop brings
together practitioners and academics to hear about the latest applied
cryptography research as well as industry problems, and is already
popular venue (with some 400 attendees and both practitioners and
academics). We believe it is strengthening the sometimes fractious community of
cryptography researchers who (want to) do more applied work. 

As part of our proposal, we will develop new methodologies for assessing applied
cryptography. In particular, by explicitly building into the design and formal
analysis process as well as empiricism. We believe this data-driven approach
will lead to better results, with theory tailored better to  the problems of
practical relevance. By interaction with the academic community via conferences,
workshops, and university visits we will both advertise this methodological
approach and gain feedback on it. 

%We hope
%that research conducted for this project will further enrich our
%colloquium and, conversely, that the presentations and discussions
%at the colloquium will stimulate the research itself.
%With this project, we will be able to expand this colloquium,
%so as to be able to invite top researchers from around the world
%somewhat more regularly.


%Education in such a dynamic field as security is continually informed by
%research developments.  Offering our graduate students experience in building
%secure and robust RNG systems, as well as assessing security of existing
%systems, will certainly benefit their education.  Our work on this project
%will also improve the classroom experience of our undergraduate students, as
%we will be better able to motivate and solve the security problems we present
%to them.  We expect courses on cryptography at Wisconsin and NYU (designed and
%taught by the PIs) to incorporate some of the notions and results of this
%project; in particular, we expect that we can devote one or more special
%topics to the use of randomness in cryptography (and specifically, the extra
%demands in the virtualized environments).

%\paragraph{Impacts on technology.} The success of this
%project should certainly be measured to a significant degree in
%terms of its impact on real-world secure systems. If the
%PIs succeed in designing practical secure PRNG systems for both virtualized and traditional environments, this will no doubt be of great utility in building secure systems. 
%
%%\ignore{
%The PIs are committed to providing open-source implementations of their PRNGs, and plan to incorporate them into existing VM systems.

%\paragraph{Interaction with industry.}
%The PIs have had regular contact with commercial and open-source
%software developers, including Microsoft, Google, VMWare (EMC), 
%and Intel. Initial discussions with industry representatives has
%revealed significant interest in our proposed research agenda. 
%We will engage industry on our research in several ways.
%As always, we will continue informal conversations at
%conferences. The PIs will also make short visits to various 
%companies. These visits will help us to get various feedback from
%industry.
%In line with our theme of robustness,  
%we will seek to obtain help identifying usage environments for which
%RNGs must perform. In line with our theme of deployability,
%we will seek input on the requirements for practical use.
%Finally, we help write new industry 
%standards for secure design
%and use of RNGs.


\paragraph{Engaging the anti-censorship community.} Our previous work on FTE has
lead us to significant interaction with, and impact on, the anti-censorship
activist community. We have frequent discussions with tool designers 
such as the Tor team, Lantern, Google's uProxy, and others. PIs Ristenpart and
Shrimpton are active members of this community and helped Google Ideas (a
philanthropic team within Google) organize a workshop in July 2014 on DPI-resistance
for anti-censorship tools. It was attended by a variety of practitioners
and academics, and had explicit goal of helping build a cohesive community in
addition to identifying research problems and potential approaches for them.

An explicit goal of our research is positive impact for activists and others who
need secure, unfettered Internet communications. Our work on FTE was
implemented in Tor and Google's uProxy. It is now included as a feature in
the Tor bundle, so activists and others can use it. 
We will pursue similar impact with the DSC work proposed here.



