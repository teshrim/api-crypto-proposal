\section{Ease of Correct Implementation as Security Property}

\paragraph{Prior Work.}

\tsnote{This thread has
the biggest risk of failure.  But also has potential for large return, as it
would change the way we think about security in the community}
\tsnote{Need to do some digging outside of crypto here!}

\todo{Rogaway's Authentication without Elision paper}
\cpnote{This is what their framework buys us. The full protocol is divided into the
``protocol core'' (PC) and the ``protocol details'' (PD). The PC is specified
by the cryptographer and the PD is specified by the engineer. The PC ``calls''
the PD in the course of the protocols execution. In the security analysis, we
assume the PD is utterly controlled by the adversary. This allows cryptographers
to ``partially specify'' a protocol that remains secure even as engineer's
requirements change; the PC must be implemented precisely, but the PD may change
to suit the programmers' needs.
%
So in this framework we're not formalizing ``implementation risk'' directly;
instead, we're proving security of protocols that have flexibility in their
implementation. It says precisely what parts you \emph{must} get right in order
to have security, and what parts are not crucial.
}

\paragraph{Tasks.}
\cpnote{We could move ``secure channels'' down here, or ``authenticated key
exchange.'' Personally I prefer leaving this under the ``API-centric crypto''
section.}
