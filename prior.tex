\section{Results of Prior NSF Support}
\label{sec:prior}


Shrimpton is the PI of NSF grant \#1319061 
``Tweakable-blockcipher-based Cryptography'', for \$433,000 and with
period 06/2013--07/2016. \todo{Add ``Distribution Sensitive Cryptography''}

There is no substantive overlap in technical
content of these proposals and the current proposal.

\heading{Intellectual Merit.} These awards have resulted in a number of results
across security and cryptography, including setting theoretical foundations for
hashing, and understanding how
to build and analyze random number generators (RNGs). The grants have funded work that
resulted in many top-tier publications, including
\cite{DRS09,BCS-journal09,OSS,BRSS10,fischlin2010random,RSS,PRS11,dyer2012peek,clrw,tct,Dyer-2013,luchaup2014libfte,luchaup2014formatted},
and two best-paper awards. 

\heading{Broader Impact.}  Prior work by PI Shrimpton has impacted a
number of diverse areas.  His line of work on hash functions influenced the designs of several of the NIST
SHA-3 entrants~\cite{DRS09,BCS-journal09,BRSS10,RSS}.  
%
Recent work uncovering significant cryptographic flaws in the IEEE
P1735 standard~\cite{CNS+17} resulted in a US-CERT vulnerability
advisory.  Earlier work~\cite{NPS} uncovered a security critical error
in the ISO 19772 standard.
%
PI Shrimptons work on format-transforming encryption FTE~\cite{Dyer-2013,luchaup2014libfte,luchaup2014formatted}
now ships with the Tor browser bundle and Google's uProxy. 
%
His work on tweakable ciphers~\cite{lrw,tct} has been incorporated into
Voltage Security's (now part of HP) product line.



%\cite{DRSS12:hmac,dodis1,dodis2,DodisRV12,DodisLWZ11,BarakDKPPSY11,}.


%\begin{newitemize}
%\item Award number: 1065134 \hspace*{3em} Amount: \$749,149 \hspace*{3em} Period of support: 9/1/2011 -- 8/31/2015 
%\item Title: ``TC: Medium: Collaborative Research: Random Number Generation and Use in Virtualized Environments''
%The work of this grant has been ongoing for one year. 
%We have so far developed significant portions
%of the theory underlying new RNGs that 
%will be deployed~\cite{DRSS12:hmac,BarakDKPPSY11,DodisRV12,DodisLWZ11}. 
%We initial results on tools to find further RNG reset vulnerabilities. 
%Two graduated masters students were partially funded by this project.
%, Benjamin Farley 
%and Thawan Kooburat, who both now work in industry on virtualization and 
%cloud computing technologies at Amazon and Facebook, respectively.
%Publications include \cite{DRSS12:hmac,dodis1,dodis2,DodisRV12,DodisLWZ11,BarakDKPPSY11}.

\ignore{
Work under these awards has resulted in a number of top-tier
publications~\cite{DRS09,BCS-journal09,OSS,BRSS10,fischlin2010random,RSS,PRS11,dyer2012peek,clrw,Dyer-2013,luchaup2014libfte,luchaup2014formatted} and
one ``Best Paper'' award. %~\cite{OSS}. 

Many of these papers are about
cryptographic hash function designs\cite{DRS,BCS-journal09,OSS,BRSS10};
\cite{DRS} and \cite{BRSS10} had impact on the designs of several of the NIST
SHA-3 entrants.  In~\cite{fischlin2010random} we explored restrictions of the random-oracle
model, and how these restrictions affect provable security.  In~\cite{RSS}, we
uncovered an important, commonly held misunderstanding about the
indifferentiability framework~\cite{RSS}, and shows how this leads to
overconfidence in the security of hash functions.  Some of the collaborations
fostered by this award have spawned new threads of research.  For example,
\cite{PRS} gives the first provable-security treatment of TLS version
1.2 (the current version), showing that it provides length-hiding authenticated
encryption.   Thinking about length-hiding led to~\cite{dyer2012peek}, which examines
the efficacy of traffic-analysis countermeasures.  In turn, this led
to early investigations into FTE~\cite{Dyer-2013,luchaup2014libfte,luchaup2014formatted}.
}

