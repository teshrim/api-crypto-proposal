\section{API-centric Cryptography}
The security community has published extensively on the \emph{usability} of
APIs~\cite{ABF+}\todo{we should have more than one citation}.  Intuitively, if
an API is easy to use correctly, developers are more likely to build secure
applications on top of it.
%
Our approach is complementary, focusing on the relationships between
functionalities that APIs present (to application developers) and the primitives
that crypto theory provides. We take the position that, in practice, APIs are
primarily driven by non-security requirements (e.g. functionality, adoption,
deployment).
%
As such, it is incumbent upon us to take an \emph{API-centric} view of the
primitives and protocols that we formalize.  In the remainder of this section,
we discuss specific areas to which we will apply this API-centric perspective.

%Note that this is a preliminary list of areas, based on prior results and current considerations.

\paragraph{Understanding an API's customers. }
It is important to recognize that there are at least two types of ``customers''
to consider when taking an API-centric viewpoint on cryptography. The first we
have already mentioned, namely application developers. Application developers
are largely concerned about functionality, and have relatively little expertise
in cryptography. They are likely to want a simple, semantically-intuitive,
high-level interface, e.g.
\begin{align*}
\varfont{SocketHandle}&\gets\algfont{SecureOpen}(\varfont{ReceiverID}), \\
\varfont{Status}&\gets\algfont{SecureSend}(\varfont{SocketHandle, Msg}),\\
\varfont{Msg}&\gets\algfont{SecureReceive}(\varfont{SocketHandle}), \\
\varfont{Status}&\gets\algfont{SecureClose}(\varfont{SocketHandle}).
\end{align*}
%
where the type and structure of the data object \varfont{Msg} would be
separately specified, e.g. it may include associated data or attributes, in
addition to the actual plaintext.
%
\cpnote{What about status? Was the connection
severed?}\tsnote{Don't understand.}
%
The second type of consumer is a security engineer. We expect security engineers
to have (at least) a reasonable understanding of basic cryptography and
cryptographic/security hygiene. Given the intricacies of
protocol design and the complexity of real systems, we expect that they would
require some degree of control over algorithmic and paremter choices.
%
\cpnote{It might be helpful to give an example of something exposed to the
``security engineer.''}\tsnote{Agreed.  So what would be the
corresponding API for security engineers?  I'm not sure, myself.  I
would think that the SE actually has an API for the components that
are needed to build up things like $\algfont{SecureSend}$, not a
version of $\algfont{SecureSend}$ that surfaces more inputs/outputs.
But if you look at that ``semantic API'' paper, the API they give to
the security engineer is very spartan, with all of the crypto actually
assumed in a Java library.  So I'm not sure what the right thing to do
is, here.  Thoughts?}
 
\todo{Need to finish out this text before leading into the task...}
\begin{task}
  We will carry out a detailed survey of existing APIs exposed by
  cryptographic libraries.  The goals will be to understand exactly
  what are the effective primitves being exported, to build a
  framework within with patterns across libraries can be established,
  to set metrics for comparing the complexity of the primitives
  exposed, and so on.  This will support multiple subsequent
  tasks.
\end{task}

\paragraph{Standardized APIs. }
\cpnote{Do you think there's too much detail here? The hedged PKE stuff has a
lot less...}\tsnote{There may ultimately be too much here, but I don't
know yet.  Anyway, it's fine to have some sections more detailed than
others, especially if some allow you to really trumpet the PI's chops.}
%
In~\cite{SSW}, PI Shrimpton gave a provable-security treatment to the design and
analysis of so-called ``cryptographic APIs'', standardized in the PKCS\#11
document\cite{xxx}.  That standard details an interface by which one may
interact with cryptographic tokens (e.g. smart cards, USB devices,
enterprise-grade HSMs) that are trusted to perform key-manangement duties,
enforcement of policies for the use of stored keys, and a limited number of
asymmetric- and symmetric-key crypto operations.  These tokens support a variety
of applications, including certification authorities, SSL/TLS acceleration, and
interbank communication.
%
\cpnote{We should mention authentication as an explicit goal
  here.}\tsnote{Go ahead!}

Our approach in this paper was API-centric, in that we took the PKCS\#11
standard as the reference point, and defined from it a new cryptographic-API
primitive.
%
\cpnote{Why ``cryptographic API primitive'' instead of ``cryptographic
  API''?}\tsnote{Because I have never liked calling the object whose
  syntax we established, and whose security notions we developed, a
  cryptographic API.  A crpytographic API exposes some effective primitive.}
%
We then formalized some of the intuitive notions of security for a
cryptographic API.  This was a complex undertaking, as one of the core
key-management functionalities is to allow the exporting and importing of keys
via ``key (un)wrapping'' operations.  (This supports, among other things, the
transportation of cryptographically protected keys between tokens.)  These keys
may have a variety of attributes associated to them that proscribe their usage,
e.g.\ this key may/may not be used to encrypt data (via a given API call), that
key may or may not be wrapped for future export, etc.  Thus, the internal state of
the key-management functionality changes as an attacker carries out permitted
API calls. (We note that works by Steele and coauthors~\cite{xxx,yyy,zzz}
exploit this changing of internal state, sometimes in concert with
underspecification in standards and device documentation, to cause catastrophic
security breaks in HSMs.) Moreover, a real adversary may be in possession of one
or more tokens ---~keep in mind that these may be physically vulnerable devices,
such as USB sticks or smart cards~--- and these are subject to corruption.  As a
result, a victim token is asked to import adverasarially-known keys.  This can
cause many other keys under management by the victim to become (effectively)
corrupted, e.g. if the attributes of the imported key allow it to be used for
wrapping.
%

While~\cite{SSW} makes important progress in the analysis of
cryptographic APIs (in particular PKCS\#11), they do leave open some
interesting avenues.  For exmaple, their handling of attributes is
quite abstract, only indicating their influence on tokens as part of
the execution model.  But what one really wants are formal mechanisms and
security notions for reasoning about the enforcement of specific
policies, e.g. that certain keys may be used only by certain entities
and only for a given set of cryptographic operations.  Also,
\cite{SSW} does not consider the public-key primitives that are
typically exposed by tokens.  Finally, it would be useful to
understand the extent to which real tokens respect PKCS\#11 and, if we
find the model of~\cite{SSW} to be significantly disconnected from
these tokens (say, because tokens often take liberties with PKCS\#11,
or because other standards are gaining traction) then we should
revisit the models.

\begin{task} We will explore these interesting and important
  extensions to~\cite{SSW}. 
\end{task}
%

\begin{task}
 \todo{This needs lead in text!}
Application developer APIs for secure send/receive of messges
(data-in-flight) and secure read/write from storage (data-at-rest).  What are
the implied primitives, and what are the security properties that developers
assume are provided.
\end{task}

\paragraph{Hedged Cryptography. }
Recent work by PI Shrimpton~\cite{BPS} revisits the theory of
hedged public-key encryption (PKE)~\cite{BBN+} from an API-centric perspective.
%
The motivation for hedged PKE is quite practical. Traditional
PKE schemes are proved secure under the assumption that they have access to a
source of uniform random bits; but in practice, PKE schemes are implemented on
systems that have faulty RNGs, or where entropy is difficult to harvest.  In
these cases, the security guarantees proved in theory are, at best, voided; at
worst, failure of the RNG can lead to recovery of the plaintext.
%
Hedged PKE is designed to satisfy traditional notions of security when provided
uniform randomness, and still deliver useful security properties as the quality
of randomness degrades.  The most elegant construction of hedged PKE works like
this~\cite{BBN+,BH15}: synthesize fresh random bits by hashing all of the
encryption inputs (the public key, the message, the provided randomness), and
then use these bits as the randomness for an underlying PKE scheme.  In
practice, implementing this simple construction is surprisingly difficult, as
the high- and mid-level APIs presented by the most commonly used crypto
libraries (e.g. OpenSSL and significant forks thereof) \emph{do not} permit one
to specify the per-encryption randomness.\footnote{Other constructions of hedged PKE that do not explicit require the
ability to manipulate the encryption coins are, likewise, difficult or
impossible to implement via existing high- and mid-level APIs.  In
their cases, it is because the primitives that are needed do not exist
in common llibraries.}  
In~\cite{BPS}, the theory of hedged
PKE is reconsidered from the perspective of what can be easily constructed given
what real APIs expose, and what provable security guarantees these can achieve.

Still, \cite{BPS} leaves open some important matters.  First of all,
the randomness sources in their models are stateless. 
Most crypto libraries export \emph{pseudorandom number generators with
  inputs}~\cite{BH05,ST15}\todo{Need to cite Dodis paper, too. You
  can grab the reference from \cite{ST15}} for providing random bits to library code and
  applications. These allow the programmer to add entropy into the PRNG's
  internal state.  
Additionally, \cite{BPS} only considers PKE.


\begin{task}
We will push hedged PKE closer to practice by exploring the
consequences of sources that are PRNGs with input.  Minimally, this will require
a reconsideration of security notions and new constructions.  We will
implement the best schemes to evaluate performance (e.g. overhead
relative to plain PKE and the schemes from~\cite{BPS}) and interact
with library maintainers to facilitate deployment of our constructions.
\end{task}

\begin{task}
We will work to hedge other primitives against faulty randomness,
using appropriate models of the randomness sources, all from the
viewpoint of what real-world APIs expose.
\end{task}


\paragraph{Secure Channels. }  One of the most basic and practically
important cryptographic functionalites is the secure channel.
Intuitively, a secure channel is the formalization
%
\ignore{
Today, TLS is deployed by ??\% of web servers and protects ??\% of
communications on the Internet.
%
\todo{Get figures.}
%
As such, this crucial code has been scrutinized
from top to bottom, including the crypto, the protocol specification,
implementations, and its deployment.\footnote{ See the \textsc{miTLS} project
for an overview: \url{https://www.mitls.org}.}
%
\ignore{
  Academic papers on the (in)security of TLS fall broadly into three categories:
  those focusing on the TLS \emph{handshake} (unilateral- or
  mutually-authenticated key exchange)~\cite{BNF+14,others}, the \emph{record
  layer} (the protocol encapsulating application data)~\cite{PRS11}, or the
  composition of the two~\cite{JKSS12,KPW13}.
}
}
Over the last decade or so, \emph{authenticated encryption with associated data}
(AEAD) has emerged as an important ingredient for constructing secure channels.
%
Its syntax is easy to reason about from a theoretical point of view, and yet is
robust enough to encompass most tasks that practitioners encounter. Moreover,
notions like \emph{nonce-misuse resistant}~\cite{RS07} and
\emph{robust}~\cite{HKR14} authenticated encryption make AEAD schemes difficult
to (inadvertently) use incorrectly. These properties make AEAD an attractive
tool for designing protocols; indeed, schemes like AES-GCM and
ChaCha20\mbox{-}Poly1305 are baked into the upcoming TLS 1.3
specification.\footnote{ See \url{https://tlswg.github.io/tls13-spec}.}

It is tempting to think that AEAD, on its own, provides a secure
channel.  \todo{I like that you are making this point.  But I think
  the writing that follows is missing the mark.  Why?  Because you
  talk about protocols and implementations, but there's no clear
  reason that any of \emph{that} provides a secure channel either!
  The point I think you are trying to make is that a secure channel
  should provide a lot more than what an AEAD scheme does, and I
  agree: so explicitly say what a secure channel is supposed to provide!
  What do applications expect from them?  Give it another try :-) And
  tighten it up, too, please.  Set up the problem, pull out the
  surprising/important nuggets, make it clear what are the directions
  they suggest, and then write a couple of clear tasks.}
 In
fact, for the thing to be useful for communicating over a network, we need to
specify a protocol in which it is used. The choice of protocol is driven largely
by engineering considerations, such as bandwidth constraints, computational
resources, and implementation complexity. However, the choice is also crucial
for security; time and time again, bugs in the \emph{protocol logic} have been
exploited to \emph{completely circumvent} the security provided by the
underlying crypto~\cite{BKN02}. These attacks point to a substantial gap between
crypto theory and the complexity of protocol logic.

We are by no means the first to point out this gap. It was recently observed by
Badertscher \etal~\cite{BMM+15} that the implicit goal of the TLS 1.3 record
layer protocol is to construct an ``augmented'' secure channel (ASC), which
provide the capability of sending a message with two parts: the first being
private, and both parts being authenticated. They emphasize that classical
security notions, such as those for AEAD ``do not capture in which contexts a
scheme satisyfying them can securely be used'' \cite[pp. 2]{BMM+15}.
Nevertheless, this work and others, such as the highly influential SSH
paper~\cite{BKN02} and the analysis the TLS record layer by PI
Shrimtpon~\cite{PRS11}, omit an important detail.

Providing a reliable (but not secure) channel between any two hosts on the Internet
is the job of the TCP/IP protocol stack. TCP in particular is designed so that
application developers need not worry about the details at all.
%
Conceptually, the API exports a stream-like functionality in which the sender
stuffs in any number of bytes and expects that they are read in order by the
sener.
%
In C, the sender sends data by invoking a function \codefont{send()}, which
takes as input its socket identifier, a buffer containing the data
\emph{fragment}, the fragment length (in bytes), and some flags; the output
indicates the number of bytes written to the channel, or an indication of
failure. The receiver invokes \codefont{recv()}, which takes as input its socket
identifier, a buffer, the maximum number of bytes to read from the stream, and
some flags. It returns the number of bytes read or an indication of failure.
%
Virtually all programming languages export a similar API for streaming data. The
alternative view (and the conventional wisdom of the cryptographic community) is
that channels provide transport of a sequence of discrete messages between two
parties. Of course, this is the true for datagram-based transport (UDP), but it
is not true for TCP.

The latter is ubiquitous for the delivery of web content; consequently,
implementations of the TLS record layer export a streaming API. In OpenSSL (the
most widely-deployed implementation of TLS), \codefont{SSL\_write()} is used to
write bytes to the stream and \codefont{SSL\_read()} is used to read bytes from
the stream. Although their syntax differes,\footnote{ See
\url{https://wiki.openssl.org/index.php/Manual:SSL_write(3)}.} these have the
same semantics as \codefont{send()} and \codefont{recv()} defined above. Thus,
the programmer can use this API just as she would use the lower-level
functionality for reliable transport, but with the added assurance that the
channel is secure.
%
Fischlin \etal~\cite{FPMG15} were the first (atleast in the cryptographic
community) to elucidate secure channels as data streams.
%
Their security notion models an adversary who transmits ciphertext ``fragments''
over the channel, and may inject and reorder fragments at will. (Concretely, we
might think of these fragments as being packets.) Their construction is based on
AEAD.

We regard the work of Fischlin \etal as an important step in an API-centric
treatment of secure channels.
%
However, there remains a wide gap between their work and cryptographic practice.
In order to apply their methodology to a real protocol, one would need to
precisely specify it as a streaming secure channel (according to their
formalism) and provide a proof of security. This on its own is a complex
undertaking, but it also places a heavy burden on the security engineers. It is
incumbent on them to implement this protocol \emph{precisely}, and if the
protocol needs to be changed (say, to address some engineering constraint), then
the cryptographer needs to provide a fresh proof.
%
\todo{Have another look at the online-AE paper. What do Fischlin \etal say
abouit?}

\begin{task}\label{task:sc}
  \cpnote{This task supports the gaol of making \emph{crypto APIs easier to use
  securely}.} \todo{Too much text within one task.  Pull out the
build-up text, then give a crisp task statement.}
  %
  A beautiful, yet largely-overlooked work by Rogaway and Stegers~\cite{RS09}
  affords an opportunity to bridge this gap. Their observation is that when
  cryptographers design protocols, they provide an abstraction that omits many
  details relevant to implementing them on real systems. (They necessarily do
  so for the reasons given above.)
  %
  They provide a framework for analyzing \emph{partially-specified protocols},
  where the ``cryptographic core'' of the protocol is separated from the
  ``protocol details'' relevant to engineers. In the analysis, one assumes the
  latter is utterly controlled by the adversary. This  allows the security
  engineer a large degree of flexibility in designing the protocol without
  compromising security.
  %
  Rogaway and Stegers apply their methodology to mutual entity authentication
  and reconsider the security of the classical NSL
  %
  \todo{Explain acronym and add a reference.}
  %
  protocol.
  %
  We propose to apply this methodology to streaming secure channels towards
  providing guidance in the design of the TLS 1.3 record layer.
\end{task}

\todo{I feel like we are missing an important task here, having to do
  with understanding what are the current APIs for secure channels,
  how those align with theoreticians' viewpoint(s) on secure channels,
looking for the ``right'' API/syntax for secure channels, i.e. one
that supports what applications expect, etc.}

\paragraph{Authenticated Key-Exchange. }
%
Task~\ref{task:sc} deals with communicants who share a key, but a crucial aspect
component of a secure channel is the initial key exchange.
%
Since the seminal work of Bellare and Rogaway~\cite{BR93}, the setting in which
authenticated key exchange is studied has not changed much; meanwhile, the needs of
real-world protocols have evolved substantially over the years. Take the TLS
handshake  as an example.
%
First, the vast majority of the literature has focused on the case of
\emph{mutual} AKE (MAKE), where both client and server wish to authenticate
the other. In fact, virtually all TLS handshakes on the Internet are only
\emph{unilaterally} authenticated (UAKE)~\cite{xxx}. (The client wishes to
verify the server's identity.)
%
Second, TLS offers features that break with the Bellare-Rogaway model
altogether. For example, Bhargavan et al.~\cite{bhargavan2014proving} point
out (a) that handshakes share states \emph{across session} (in
Bellare-Rogaway each session has its own state), and that (b) the same
material may be used by the same entity in \emph{different protocols}
(requires the underlying algorithms be \emph{agile}, a
la~\cite{acar2010cryptographic}).
%
Third, an important consideration in the upcoming TLS 1.3
specification~\cite{tls13} is \emph{session resumption}, where the goal is
to piggy-back a fresh key exchange on the data channel between a client and
server who have already performed the full handshake. This feature is
crucial from an engineering perspective, but as is well-known~\cite{xxx}, we
cannot provide the same level of security for the message on which we are
piggy-backing the exchange as is possible for subsequent messages.

These issues (and a myriad of others) point to a gap between how programmers
expect they should be able to use cryptography, and how cryptography is to
be used securely. This presents an opportunity to distill from design
patterns in existing KE protocols (TLS~\cite{xxx}, IPSec~\cite{xxx}, Signal~\cite{xxx}, Noise~\cite{xxx}, etc.) an
abstraction (and corresponding API) that cleanly captures the needs of
software engineers.
  %
  A promising direction is to begin with Dodis and
  Fiore~\cite{dodis2017unilateral}, who show that security in this setting is
  much easier to model than the more general MAKE setting of Bellare-Rogaway.
  An approach by Krawczyk~\cite{krawczyk2016unilateral-to-mutual} can be
  employed to transform a UAKE scheme to a MAKE scheme.
  %
  Dodis and Fiore define a natural primitive they call
  \emph{comfirmed encryption} for KEM-based UAKE, which has an API that, we
  feel, is intuitive for security engineers.
  %
  A limitation, however, is that comfirmed encryption cannot provide forward
  secrecy on its own. 

\begin{task}
%\cpnote{\emph{Propose an API} for a field that needs one.}
We will elucidate \tsnote{``Elucidate'', or establish?} the syntax and
security for DH-based UAKE whose API is similarly
intuitive. 
\end{task}

Going in a different direction, Rogaway and Stegers~\cite{RS09} 
  suggest their methodology could be extended from entity
  authentication to AKE. We find it somewhat remarkable that this task
  has not been undertaken, and we propose to do so. \todo{This needs a
    bit more lead-in.}
\begin{task}
%\cpnote{Make it easier to use APIs correctly.}
We will apply the partially-specified-protocol methodology of~\cite{RS09} to authenticated key-exchange.
  %
\end{task}

\todo{Is there more here?  What about the ideas that
  Trevor Perrin suggested, specifically for AKE?  I feel like we are
  missing a chance to build in more tasks and show more vision}