\section{API-centric Cryptography}
The security community has published extensively on the \emph{usability} of
APIs; recent developments in this area include \cite{ABF+,HHL+17,IKND16}.
Intuitively, if an API is easy to use correctly (and hard to use
\emph{incorrectly}), developers are more likely to build secure applications on
top of it.
%
Our approach is complementary, focusing on the relationships between
functionalities that APIs present and the primitives that crypto theory
provides. In practice, API design is primarily driven by non-security
requirements (e.g. functionality, adoption, deployment).
%
As such, we suggest to take an
\emph{API-centric} view of the primitives and protocols that we
formalize.

We are not the first to recognize the benefits of this viewpoint.
Recent works by Fischlin \etal~\cite{FPMG15} and Huang
\etal~\cite{HRRV15}, for example, make clear efforts to mimic actual
APIs in their formalizations.  Quoting~\cite{HRRV15}, ``To
accurately model the underlying goal... we adopt an \emph{API-based
  view}, where the segmentation of a plaintext is determined by the
caller.'' (Our emphasis.)  What we propose is to fully embrace this
viewpoint, and to apply it broadly to cryptography.

In the remainder of this section, we discuss some specific areas to which we will apply
this API-centric perspective.
%
%\cpnote{I think we need to define \emph{API-centric} in the intro.}

%Note that this is a preliminary list of areas, based on prior results and current considerations.

\paragraph{Understanding the Landscape. }
It is important to recognize that there are at least two types of ``customers''
to consider when taking an API-centric viewpoint on cryptography. The first we
have already mentioned, namely \emph{application developers}. Application developers
are primarily concerned about functionality, and typically have little
to know real expertise in cryptography.  (Nor should they, frankly.)
%
%\cpnote{We could say something stronger: that the needn't have \emph{any}
%expertise in cryptography, and merely understand that the connection needs to be
%secure (in some sense).}
%
For them, the interface to cryptographic functionalities should be
simple and have intuitive semantics, e.g.,
\begin{align*}
\varfont{SecureSocketHandle}&\gets\algfont{SecureOpen}(\varfont{ReceiverID}), \\
\varfont{Status}&\gets\algfont{SecureSend}(\varfont{SecureSocketHandle, Msg}),\\
  (\varfont{Msg}, \varfont{Status}) &\gets\algfont{SecureReceive}(\varfont{SecureSocketHandle}), \\
\varfont{Status}&\gets\algfont{SecureClose}(\varfont{SecureSocketHandle}).
\end{align*}
%
as an API for securely exchanging messages with an specified partner.
The API would also specify the structure of the inputs and outputs.
For example, the \varfont{Msg} object may include associated data or attributes, in addition
to the actual plaintext.
%
%The object \varfont{Status} indicates the channel's health. For example, the
%channel may have been torn down by the peer, or the destination address may be
%unreachable.


The above API neatly hides an enormous amount of plumbing implemented by the
\emph{security engineer}. There are a number of components to put together. The
high-level tasks are to authenticate the peer, exchange a key, use the key to
transmit the message, and securely tear down the session. Designing and
implementing such a protocol gets into the weeds quite quickly. This is an area
where having good APIs for things like digital signatures, message
authentication codes, and encryption is extremely helpful; even so, it is rarely
apparent how these are to be securely composed. This difficulty is compounded by
the fact that the protocol needs to be as efficient as possible, as application
developers (and users of the applications in turn) are generally unwilling to
pay a heavy price for security.
%
To illustrate these issues, consider the following API for
unilaterally\footnote{By unilaterally, we mean that the party opening the
channel authenticates the peer, but not the other way around.}
authenticated key exchange:
\begin{align*}
  \varfont{SocketHandle} &\gets\algfont{Open}(\varfont{ReceiverAddr}), \\
  (\varfont{Key}, \varfont{Status}) &\gets \algfont{NegotiateKey}(
    \varfont{SocketHandle}, \varfont{ReciverCert}, \varfont{CertAuth})
\end{align*}
where $(\varfont{ReceiverAddr}, \varfont{ReceiverCert})$ comprise
\varfont{ReceiverID} (above).  First, note that the root-of-trust,
\varfont{CertAuth}, is implicit in the first API. Unless we are to trust the
developer to manage certificates (as many existing APIs do), then
the security engineer needs to make this transparent to upper levels. This has
inherent risks, which may not be obvious to the security engineer~\cite{FHM+12}. Second,
there are numerous ways to realize \algfont{NegotiateKey}, each with varying
degrees of security and efficiency. Third, modern protocols, such as TLS, are
designed so that application data may be piggy-backed on the handshake. (This is
called 0-RTT, or zero round trip time, data.) As such,
providing an opaque interface is likely not desirable. In general, the ``right''
level of exposure to the application developer or security engineer is highly
task-dependent.
%
\if{0}
  (Consider for example, the well-known deficiency in TLS version $1.2$ (and
  older); The final \textsc{Finish} message being MACed renders it impossible to
  prove security in the key-indistiguishiblity sense, which means the handshake
  is not directly composable with the record protocol.)
\fi
%
\if{0}
\cpnote{It might be helpful to give an example of something exposed to the
``security engineer.''}\tsnote{Agreed.  So what would be the
corresponding API for security engineers?  I'm not sure, myself.  I
would think that the SE actually has an API for the components that
are needed to build up things like $\algfont{SecureSend}$, not a
version of $\algfont{SecureSend}$ that surfaces more inputs/outputs.
But if you look at that ``semantic API'' paper, the API they give to
the security engineer is very spartan, with all of the crypto actually
assumed in a Java library.  So I'm not sure what the right thing to do
is, here.  Thoughts?}
\todo{Need to finish out this text before leading into the task...}
\fi

\begin{task}
  We will carry out a detailed survey of existing APIs exposed by
  cryptographic libraries. The goals will be to understand exactly
  what are the effective primitives being exported, to build a
  framework for understanding design patterns common across libraries,
  to set metrics for comparing the complexity of the primitives
  exposed, and so on.
\end{task}
\noindent
The above will support multiple subsequent tasks.  We note that the
\emph{effective} cryptographic primitives that are presented to the application
developer may be considerably different that those considered cryptographers.

\begin{task}
Beginning with application developer APIs for secure exchange of messages
and secure read/write from storage, we will formalize the
cryptographic primitives they imply, and develop security notions that
capture the security properties that developers assume are provided.
(See also the related Task 3.)
\end{task}



\paragraph{Secure Channels. }
Perhaps the most practically important cryptographic functionality is the
secure channel, the task of which is to provide privacy and authenticity for
message flows between peers.
%
Over the last decade or so, \emph{authenticated encryption with associated data}
(AEAD) has emerged as an important ingredient for constructing secure channels.
%
Its syntax is easy to reason about from a theoretical point of view, and yet is
robust enough to encompass most tasks that practitioners encounter. Moreover,
notions like \emph{nonce-misuse resistant}~\cite{RS06} and
\emph{robust}~\cite{HKR15} authenticated encryption make AEAD schemes difficult
to (inadvertently) use incorrectly. These properties make AEAD an attractive
tool for designing protocols.

It is tempting to think that AEAD, on its own, provides a secure channel.  But
as pointed out by Badertscher \etal~\cite{BMM+15}, classical security
notions, such as those for AEAD ``do not capture in which contexts a scheme
satisfying them can securely be used.''  Indeed, bugs in the
\emph{protocol logic} have been exploited to circumvent the
security provided by the underlying
crypto~\cite{Vau02,BKN02,APW09}.\footnote{See
\url{https://www.mitls.org/pages/attacks\#protocol} for a list of
protocol-level attacks on TLS.}  This gap has spurred the study of
\emph{stateful} authenticate encryption schemes that model secure channels more
closely, such as the highly influential SSH paper~\cite{BKN02} and the analysis
of the TLS $1.2$ record layer by PI Shrimpton and coauthors~\cite{PRS11}. Boldyreva
\etal~\cite{BDPS12} later extended~\cite{BKN02} to encompass the protocol-level
attack of~\cite{APW09} that exploits fragmented delivery of the
ciphertext stream. 

Fischlin \etal~\cite{FPMG15} observe that this line of work still
misses an important point about the APIs (i.e. the effective cryptographic
primitives) that are presented in practice.  Namely, the prior work treated
channels as providing secure transport of atomic messages between two parties.
%
%\cpnote{This is a bit redundant in the context of the first section
%``understanding an API's customers. Maybe we should pair this down a bit?}
%
%Providing a reliable (but not secure) channel between any two hosts on the
%Internet is the job of the TCP/IP protocol stack. TCP in particular is designed
%so that application developers need not worry about the details at all.
But TCP is ubiquitous for the delivery of web content, and virtually
all programming languages export an API for \emph{streaming data} over
TCP.  Likewise, implementations of
the TLS record layer also export a streaming data API. 
%
\ignore{
In OpenSSL (the most widely-deployed implementation of TLS), \codefont{SSL\_write()} is used to write
bytes to the stream and \codefont{SSL\_read()} is used to read bytes from the
stream.  An application developer can use this API just as she would use the
lower-level functionality for reliable transport, but with the added assurance
that the channel is secure.
}
Fischlin \etal~\cite{FPMG15} were the first to give a cryptographic
treatment of secure channels as data streams. (Prior
work~\cite{BDPS12,BFK+13} treated plaintexts and/or
ciphertexts as atomic.)
%
Their syntax directly captures a streaming API, and their security notions
(privacy and authenticity) model an adversary who transmits ciphertext
``fragments'' over the channel, and may drop, inject, and reorder fragments at
will.  This allows them to model streaming protocols (such as TLS) that divide
the message stream into discrete ``records'' and encrypt each individually
before transmitting. The peer may receive only a piece of a record, or multiple
records simultaneously.

We regard~\cite{FPMG15} as an important step in an API-centric
treatment of secure channels, but a number of important practical
issues remain.
%
In order to apply their methodology to a real protocol, one would need to
precisely specify it as a streaming secure channel (according to their
formalism) and provide a proof of security. This on its own is a complex
undertaking (cf.~\cite{DLFK+17}), but it also places a heavy burden on security
engineers. It is incumbent on them to implement this protocol \emph{precisely},
and if the protocol needs to be changed (say, to address some engineering
constraint), then the cryptographer needs to provide a fresh proof.

A largely-overlooked work by Rogaway and Stegers~\cite{RS09}
affords an opportunity to bridge this gap.\footnote{Degabreile
  \etal~\cite{DPW11} also note this: ``this is an intriguing approach
  that we expect to see further developed.''} Their observation is that when
cryptographers design protocols, they provide an abstraction that omits many
details relevant to implementing them on real systems.
%
This is not without reason, of course; the needs of systems are complex and always
changing.
%
TLS is a prime example. The record layer is the piece of the protocol
encompassing all messages between client and server. The specification document
for the upcoming version 1.3~\cite{tls13} is comprised of a rich set of rules
for how the stream of messages are mapped to records: some types of messages may
be coalesced into a single record, others must be the only message in a record,
and so on. These rules may seem, at first glance, irrelevant from a security
perspective, but it quickly becomes clear that the analysis must address
them.
%
\if{0}{Transmission of early data was added to the protocol in
order to address a real-world problem: that the TLS handshake incurs lots of
network latency, which makes it difficult to implemented at scale. This lead to
the ``False Start'' extension for TLS 1.2~\cite{RFC7918}, and 0-RTT data will be
a core component of TLS 1.3. But these extensions adds complexity and render the
analysis more difficult.}\fi
%
Rogaway and Stegers provide a framework for analyzing \emph{partially-specified
protocols}, where the ``cryptographic core'' of the protocol is separated from
the ``protocol details'' relevant to engineers. In the analysis, one assumes the
latter is utterly controlled by the adversary. Thus, a proof of security under
this pessimistic assumption affords the protocol designer a high degree of
flexibility, while providing a certain amount of robustness to bugs in
implementations.

\begin{task}\label{task:sc}
  We propose to apply Rogaway-Stegers methodology to streaming secure channels
  towards providing guidance in the design of the TLS 1.3 record layer.
\end{task}


\if{0}{
  \cpnote{Fischlin \etal say that OAE is unrelated to their work, although they
  didn't read~\cite{HRRV15}. (Their OAE citations are prior work.)
  Nonetheless, I tend to agree with them. First, the syntax is sufficiently
  different. Second, the goals are different; \cite{HRRV15} want to model
  \emph{online} and \emph{constant-space} encryption/decryption;
  in~\cite{FPMG15} neither of these are mandated.}\todo{Chris: is there actual
  text to add from your note?} \cpnote{Nope. This was just a response to one of
  your earlier comments.}
}\fi
%\todo{I feel like we are missing an important task here, having to do
%with understanding what are the current APIs for secure channels,
%how those align with theoreticians' viewpoint(s) on secure channels,
%looking for the ``right'' API/syntax for secure channels, i.e. one
%that supports what applications expect, etc.}
%
%\cpnote{To me this is already done. \cite{FPMG15} syntax is quite close to real
%APIs. The syntax I've worked out for this paper is \emph{even closer}, but I
%don't think that's the main contribution of the paper. The main contribution is
%the security model.}

\paragraph{Authenticated Key Exchange. }
The key agreement protocol used to establish the channel is also amendable to an
API-centric treatment.
%
Since the seminal work of Bellare And Rogaway~\cite{BR93}, the vast majority of
the literature on authenticated key exchange has focused on the case of
\emph{mutual} AKE (MAKE), where both client and server wish to authenticate the
other. In fact, virtually all TLS handshakes on the Internet are only
\emph{unilaterally} authenticated (UAKE). (The client wishes to verify the
server's identity.)
%
\if{0}{ % We're not actually dealing with this.
Second, TLS offers features that break with the Bellare-Rogaway model
altogether. For example, Bhargavan et al.~\cite{bhargavan2014proving} point
out (a) that handshakes share states \emph{across session} (in
Bellare-Rogaway each session has its own state), and that (b) the same
material may be used by the same entity in \emph{different protocols}
(requires the underlying algorithms be \emph{agile}, a
la~\cite{acar2010cryptographic}).
}\fi
%
We alluded to another issue above. An important consideration in the upcoming TLS 1.3
specification~\cite{tls13} is the transmission of 0-RTT data.
%
This feature is crucial from an engineering perspective, but as is
well-known~\cite[Section 2.2]{tls13}, we cannot provide the same level of
security for the message on which we are piggy-backing the exchange as is
possible for subsequent messages. In particular, \emph{forward security} is out
of reach.

These issues point to a gap between how programmers expect they should be able
to use cryptography, and how cryptography is to be used securely. This presents
an opportunity to distill from design patterns in existing KE protocols (e.g.,
the TLS handshake) an abstraction (and corresponding API) that cleanly captures
the needs of software engineers.
%
A promising direction is to begin with Dodis and
Fiore~\cite{dodis2017unilateral}, who show that UAKE is much simpler to model
than MAKE, which requires a more general protocol framework, such as
Bellare-Rogaway.
%
Dodis and Fiore define a natural primitive they call
\emph{confirmed encryption} for KEM-based UAKE, which has an API that, we
feel, is intuitive for security engineers. Briefly, the sender encrypts its
message with the public key of the recipient essentially as one would with a
conventional encryption scheme. The receiver decrypts the ciphertext, then
transmits a ``confirmation'' proving to the sender that the message was
received.
%
This primitive immediately yields UAKE. Using an approach by
Krawczyk~\cite{krawczyk2016unilateral-to-mutual}, it can also be used as a
building block for a MAKE protocol.

Distilling a complex problem in this way is invaluable to practice. A
limitation of~\cite{dodis2017unilateral}, however, is that confirmed encryption
cannot provide forward security on its own. As this has become a first-class
concern, we feel it deserves a first-class primitive.
%
%\cpnote{Tom: I kind of stole the language of the last line from you.}

\begin{task}
  Extending the work of~\cite{dodis2017unilateral}, we will study UAKE protocols
  that provide \emph{forward security} with the aim of providing an API that is
  intuitive for security engineers.
\end{task}

\noindent
Having such an API would simplify the task of designing real-world KE
protocols, but it is only a component of the design process.

Going in a different direction, Rogaway and Stegers~\cite{RS09} suggest their
methodology can be extended from entity authentication to AKE.
Perhaps due to the limited attention this paper has received in the
cryptographic community, this natural extension as not been explored.
\begin{task}
  %\cpnote{Make it easier to use APIs correctly.}
  We will apply the partially-specified-protocol methodology of~\cite{RS09} to
  authenticated key-exchange.
  %
\end{task}


%\todo{Is there more here?  What about the ideas that
%  Trevor Perrin suggested, specifically for AKE?  I feel like we are
%  missing a chance to build in more tasks and show more vision}
%\cpnote{I haven't come up with anything.}

\paragraph{Standardized APIs. }
%\cpnote{Do you think there's too much detail here? The hedged PKE stuff has a
%lot less...}
%\tsnote{There may ultimately be too much here, but I don't
%know yet.  Anyway, it's fine to have some sections more detailed than
%others, especially if some allow you to really trumpet the PI's chops.}
%
In~\cite{SSW}, PI Shrimpton and coauthors gave a provable-security treatment to
the design and analysis of so-called ``cryptographic APIs'', standardized in the
PKCS\#11 document.\footnote{See
\url{https://www.oasis-open.org/committees/tc_home.php?wg_abbrev=pkcs11} for the
latest spec.} That standard details an interface by which one may interact with
cryptographic tokens (e.g. smart cards, USB devices, enterprise-grade HSMs) that
are trusted to perform key-management duties, enforcement of policies for the
use of stored keys, and a limited number of asymmetric- and symmetric-key crypto
operations.  These tokens support a variety of applications, including entity
authentication, certification authorization, SSL/TLS acceleration, and interbank
communication.
%

Our approach in this paper was API-centric, in that we took the PKCS\#11
standard as the reference point, and defined from it a new cryptographic-API
primitive.
%
%\cpnote{Why ``cryptographic API primitive'' instead of ``cryptographic
%  API''?}\tsnote{Because I have never liked calling the object whose
%  syntax we established, and whose security notions we developed, a
%  cryptographic API.  A cryptographic API exposes some effective primitive.}
%
We then formalized some of the intuitive notions of security for a
cryptographic API.  This was a complex undertaking, as one of the core
key-management functionalities is to allow the exporting and importing of keys
via ``key (un)wrapping'' operations.  (This supports, among other things, the
transportation of cryptographically protected keys between tokens.)  These keys
may have a variety of attributes associated to them that proscribe their usage,
e.g.\ this key may/may not be used to encrypt data (via a given API call), that
key may or may not be wrapped for future export, etc.  Thus, the internal state of
the key-management functionality changes as an attacker carries out permitted
API calls. (We note that works such as~\cite{KS13} exploit this changing of
internal state, sometimes in concert with underspecification in standards and
device documentation, to cause catastrophic security breaks in HSMs.) Moreover,
a real adversary may be in possession of one or more tokens ---~keep in mind
that these may be physically vulnerable devices, such as USB sticks or smart
cards~--- and these are subject to corruption.  As a result, a victim token is
asked to import adverasarially-known keys.  This can cause many other keys under
management by the victim to become (effectively) corrupted, e.g. if the
attributes of the imported key allow it to be used for wrapping.
%

While~\cite{SSW} makes important progress in the analysis of
cryptographic APIs (in particular PKCS\#11), they do leave open some
interesting avenues.  For example, their handling of attributes is
quite abstract, only indicating their influence on tokens as part of
the execution model.  But what one really wants are formal mechanisms and
security notions for reasoning about the enforcement of specific
policies, e.g. that certain keys may be used only by certain entities
and only for a given set of cryptographic operations.  Also,
\cite{SSW} does not consider the public-key primitives that are
typically exposed by tokens.  Finally, it would be useful to
understand the extent to which real tokens respect PKCS\#11 and, if we
find the model of~\cite{SSW} to be significantly disconnected from
these tokens (say, because tokens often take liberties with PKCS\#11,
or because other standards are gaining traction) then we should
revisit the models.

\begin{task}
  We will explore these interesting and important extensions
  to~\cite{SSW}: (1) formal mechanisms and
security notions for reasoning about the enforcement of specific
policies, (2) incorporating the public-key primitives that are
typically exposed by tokens, (3) establishing how closely tokens
adhere to PKCS\#11 and revisiting the formal models based on what we find.
\end{task}
%


%%%%%%%%%%%%%%%%%%%%%%%%%%%%%%%%%%%%%%%%%%%%%%
\ignore{
\paragraph{Ease of Correct Implementation as Security Property. }

%\paragraph{Prior Work.}
\todo{Write.}\tsnote{Chris, at your suggestion, I've downgraded this
  former section to a paragraph-subsection within API-centric
  crypto. Let's see how it works.}

\tsnote{May need to do some digging outside of crypto here!}

\todo{Rogaway's Authentication without Elision paper}
\cpnote{This is what their framework buys us. The full protocol is divided into the
``protocol core'' (PC) and the ``protocol details'' (PD). The PC is specified
by the cryptographer and the PD is specified by the engineer. The PC ``calls''
the PD in the course of the protocols execution. In the security analysis, we
assume the PD is utterly controlled by the adversary. This allows cryptographers
to ``partially specify'' a protocol that remains secure even as engineer's
requirements change; the PC must be implemented precisely, but the PD may change
to suit the programmers' needs.
%
So in this framework we're not formalizing ``implementation risk'' directly;
instead, we're proving security of protocols that have flexibility in their
implementation. It says precisely what parts you \emph{must} get right in order
to have security, and what parts are not crucial.
}

}
