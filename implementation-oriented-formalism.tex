\section{Implementation-considerate Formalisms}
\label{sec:formalisms}
The modern ``provable security'' approach to cryptography loosely follows a
three-step recipe: define a precise syntax for the primitive under study, define
a formal notion of security for that primitive, realize the primitive,
and prove it meets the security notion. The first step, defining the syntax,
essentially defines the component objects that collectively make up the
primitive, e.g. ``An encryption scheme $\Pi$ is a triple of algorithms
$(\calK,\calE,\calD)$ ...''  The syntax typically describes the number and type
of the inputs and outputs of these components, too, e.g. ``The decryption
algorithm provides a mapping $\calD\colon\bits^{k}\times\bits^* \to \bits^* \cup
\{\bot\}$ ...'', as well as requirements on the behavior of these components in
any correct realization of them.

%\cpnote{This is great.}
%
The syntax for a new primitive provides an API for researchers. For
those wishing to \emph{realize} the primitive (in theory), it makes clear what
are the functionalites that must be instantiated, and what are the assertions that
must hold for any realization; for those wishing to \emph{use} the primitive, it
makes clear what calls will be available to them, and what are the expected
inputs and outputs of those calls.
%
Like a real software API, syntax is considered good if it serves the needs of
its ``customers''; in this case, if it supports clear security notions,
facilitates the writing (and verification) of proofs, and is easily built
upon.\footnote{As an aside, PI Shrimpton has found the analogy of
\emph{syntax-as-API} as a useful pedagogical tool. Computer science students
often seem far more familiar and comfortable with programming abstractions than
with mathematical ones.}

%Of course, this analogy only goes so far.  When a researcher argues for a change
%to the accepted syntax ---~a new API for a primitive~--- there is no real cost
%incurred if it is adopted.  Prior works using the old syntax do not
%suddenly``break''; in the worst case they may (eventually) be viewed as
%outdated.  If anything, the new syntax presents an opportunity for researchers
%revisit prior work and publish updated findings.

%\cpnote{I think you should add a nonce to the AEAD syntax below, as I think is
%usual. This also gives you an opportunity to discuss the issue with nonces in
%practice: that there's not always an obvious choice for a nonce, programmer's
%don't know how/why to use them, etc.}
%
Of course, there is a considerable gap between the formal syntax that papers
provide and a \emph{real} API, i.e., the functionalities that need to be
implemented. For example, the well-established syntax for symmetric-key
decryption with associated data is a deterministic algorithm $\calD\colon\calK
\times \calH \times \bits^* \to \bits^* \cup \{\bot\}$, where $\calK$ and
$\calH$ are sets. We cryptographers would typically write something like $M
\gets \calD^H_K(C)$ for the decryption of ciphertext~$C\in\bits^*$, with
associated data~$H \in \calH$, under key~$K \in \calK$.  But this quietly
assumes a number of implicit functionalities that must be attended to in
practice.

First of all, the formal syntax assumes that~$H$ and~$C$ are simply presented as
inputs. In reality the decryption process may need to parse (aka deserialize,
unmarshall) what is received from the channel into the ciphertext~$C$ and any
context information (aka associated data) that was sent with~$C$. (For example,
TLS records encode the length of the cipheretxt.) Moreover, the overall
decryption process may depend on both the context that is associated to this
ciphertext transmission \emph{and} on context that the receiver maintains for
the channel. (Note that the syntax is silent as to the source of the associated
data; it may have been transmitted with~$C$ or made available locally.) In
practice the receiver may maintain local message counters, lists of nonces
already observed on the channel, information about the overall status and health
of the channel (e.g. is it still considered active), and so on.  The decryption
key~$K$ is also part of the context that the receiver maintains.

Next, the formal syntax says that decryption returns a
string or a distinguished symbol~$\bot$, the latter being traditionally
understood by cryptographers to mean ``invalid''.  In practice, this means that
some logic must be implemented to decide (in)validity, and that when the
decision is ``invalid'' some error-handling mechanism ensures that the
decryption process returns nothing but a distinguished error message.

\begin{figure}[t]
  \begin{center}
    \fpage{0.58}{
      \underline{\algfont{DECRYPT}(\rwordfont{var} \varfont{ChannelContext}, \varfont{Msg}):}
      \begin{align*}
        &(\varfont{SentContext},\varfont{Ciphertext}) \gets\algfont{Deserialize}(\varfont{Msg})\\
        &(\varfont{ExternalVals, InternalVals}) \gets \algfont{Decrypt}(
            \rwordfont{var}\;\varfont{ChannelContext},\varfont{SentContext}, \varfont{Ciphertext})\\
        &\varfont{Verdict} \gets \algfont{IsValid}(\varfont{InternalVals})\\
        &\varfont{ExternalVals} \gets
                                    \algfont{ErrHandler}(\rwordfont{var}\;\varfont{ChannelContext},
                                    \varfont{Verdict}, \varfont{ExternalVals})\\
        &\varfont{Return} \varfont{ ExternalVals}
      \end{align*}
    }
  \end{center}
%
\caption{Describing symmetric-key decryption of received message~\varfont{Msg} in terms of
  explicit functionalities that must be realized. All variables are passed by
  value except those preceded by the keyword \rwordfont{var}. The
  \varfont{ChannelContext} is receiver-maintained information about the channel,
  and may contain values shared with the sender (e.g., the key, channel ID),
  along with local state.  The \varfont{SentContext} is context data associated
  to this particular ciphertext. The \varfont{ExternalVals} include the message
  and are intended to be released to the external caller, and the
  \varfont{InternalVals} are intended for use only within the decryption process
  boundary.  The types of all input and output values are implicit.}
\label{fig:syntax-api-example}
\end{figure}
%
In Figure~\ref{fig:syntax-api-example}, we give an example of a more
API-like presentation of decryption.
%
Making these implied functionalities more explicit has several benefits.  The
most obvious is helping developers to see what they need to implement, rather
than assuming they will be able to correctly tease out what's needed.
Conversely, it prompts the theoretician to think in terms of abstractions that
appear in practice.
%
For both developer and theoretician, this style of presentation can surface a
clearer picture of what security demands of these functionalities. For example,
specifying error handling as an explicit functionality\footnote{Following
Rogaway and Stegers~\cite{RS09}, variables are passed by value except where
annotated by \rwordfont{var}. In this case they are passed by reference so that
their value may be modified by the caller.}
\[
\varfont{ExternalVals} \gets
                                  \algfont{ErrHandler}(\rwordfont{var}\;
                                  \varfont{ChannelContext},
                                  \varfont{Verdict}, \varfont{ExternalVals})
\]
more naturally prompts one to consider questions, like, what \varfont{ExternalVals} should
be returned for a given \varfont{Verdict}, and what
security-relevant information might be leaked by them?  How should the
\varfont{ChannelContext} be updated for a given \varfont{Verdict}?


In Figure~\ref{fig:EtM-aead} we give two potential realizations of the
functions needed for decryption. In particular, they specify the components of
Figure~\ref{fig:syntax-api-example} used to realize \algfont{DECRYPT}. The
underlying AEAD scheme follows the conventional encrypt-then-MAC paradigm.
%
On the left side of the figure, we give a realization of conventional
encrypt-then-MAC AEAD, following what we
theoreticians consider as best practice.  Namely, if the tag check fails, then
one should label the ciphertext as invalid, and suppress all output other than
the (single) error message ``Invalid'' (or $\bot$ in typical syntax).
Additionally, when ``Invalid'' is returned, the channel status field
of \varfont{ChannelContext} is changed to ``closed''.
%
On the right side, we give a more complex encrypt-then-MAC style realization that includes
in the \varfont{ChannelContext} a list of valid tokens that are used within
the decryption process. It also allows for a stream ID (sid) as part
of the transmitted AD (i.e. the \varfont{SentContext}).  In the
absence of any decryption errors, the decryption process will return
(as the output of \algfont{ErrHandler}) the stream ID and the
plaintext.  If the tag check fails, then the
\varfont{Verdict} that is returned by \algfont{IsValid} is either
``Invalid'', or $\langle\mbox{"Invalid"},\mbox{"BadToken"}\rangle$,
depending on whether or not the token ID (tid) recovered within
\algfont{Decrypt} is in the token list.  In either case, the
decryption process returns ``Invalid'' and the channel status field
of \varfont{ChannelContext} is changed to ``closed'', as on the left.
If the tag check \emph{succeeds} but the recovered tid is invalid,
then the decryption process still returns ``Invalid'', but the
\varfont{ChannelContext} is not changed.
\todo{Tom: finish... one can imagine that this might be a real-world
  behavior, where a ciphertext validity failure is fatal to the
  channel, but a bad token is considered less of an infraction so the
  channel is not closed.  (Say, if the cost to tear down/set up a new
  channel is considered unreasonable for the given application, so you
  only want to do it if things are really bad.)  Maybe in this case
  the application requests retransmission, or sends an
  \emph{application layer} error message back through the channel.} 

\tsnote{Say that this may seem like overkill for ``simple'' primitives
like symmetric-key decryption, but that (1) it only seems that way
because we are already implicitly thinking about what it means for
things to be valid, etc., and (2) we ``know'' not to do certain things
because mistakes were made in the past, by people who didn't know at
the time --- hindsight is 20-20.}

\begin{task}
{We will reconsider the traditional formalisms for cryptographic primitive from an implementation considerate perspective, and establish security notions with respect to the new syntax. }
\end{task}

\tsnote{Need to address head-on the complaint that this is going to
  make for more complicated analysis.}

Let us consider a concrete example to illustrate this approach.
%
As we have already alluded to, it is common in cryptography to think of
decryption as either returning a message or~$\bot$, indicating that something
went wrong. This choice implies that if the output is not $\bot$, then we must
assume it is valid.
%
This is a ubiquitous choice in cryptography that is rather strange from an API
perspective. Virtually all APIs that we are aware of separate the message and
error outputs. Errors also have varying degrees of severity: some are
\emph{fatal}, requiring the connection be torn down; some are simply
\emph{warnings} that leave to the application the decision to process the
message or to tear down the connection; and some might completely benign,
indicating, e.g., that more data is available for reading, or that the
end-of-file has been reached.
%
Some works consider a richer semantics for the error~\cite{BDPS13,FPMG15}, but
even these couple the message and error so that one of these is output, but
never both.
%
We are aware of no work that explicitly separates the message and error. Of
course, the impact of this separation on security is highly dependent on the
setting in which it is used.

\begin{task}\label{task:sc}
  In our study of secure channels (\textbf{task~\ref{task:sc}}), we will explore the implications of
  separating the message and error in the output of decryption.
\end{task}

\ignore{
\paragraph{Prior Work. }
%
\todo{Look at Rogaway's online-AE paper.  I seem to remember him
  saying that the main contribution of that paper was API oriented}
\cpnote{\cite[Section 5]{HRRV15}: ``We provie a new notion for online-AE ... To
accurately model the underlying goal, not only must the security definition
depart from [prior work], but so too must a scheme's basic syntax. In
particular, we adopt an API-based view, where the segmentation of a plaintext is
determined by the caller.''}
}


%%%%%%%%%%%%%%%%%%%%%%%%%%%%%%%%%%%%%%%%%%%%%%%%%%%%%%5

\begin{figure}
\centering
\hfpages{0.475}
{
Given \varfont{ChannelContext}=\\[-0.5ex]
\nudge$\langle$ModeKey $K$, PRFKey $L$, ChannelStatus
``open''$\rangle$ 
%Given $\varfont{Msg} =\langle V,H,C \rangle$

\medskip
Alg.\ \algfont{Deserialize}(\varfont{Msg}):\\
\nudge $V \gets \varfont{Msg}.V$; $H \gets \varfont{Msg}.H$; $C \gets
\varfont{Msg}.C$\\
\nudge Return $\left((V, H),C\right)$

\medskip
Alg.\ \algfont{Decrypt}({\bf var} \varfont{ChannelContext},$(V,H)$, $C$):\\
\nudge Parse $C$ into tag $T$ and remainder $Z$\\ 
%\nudge $K \gets \varfont{ChannelContext.ModeKey}$\\
%\nudge $L \gets \varfont{ChannelContext.PRFKey}$\\
\nudge if $F_L(V,A,Z) \neq T$ then \varfont{InternalVals} $\gets$ ``TagFail''\\
\nudge\varfont{ExternalVals} $\gets \calD_K^{V}(Z)$\\
\nudge Return (\varfont{ExternalVals},\varfont{InternalVals})

\medskip
Alg.\ \algfont{IsValid}(\varfont{InternalVals}):\\
\nudge \varfont{Verdict} $\gets$ ``Valid''\\
\nudge if \varfont{InternalVals} =  ``TagFail'' then \varfont{Verdict} $\gets$ ``Invalid''\\
\nudge Return \varfont{Verdict}

\medskip
Alg.\ \algfont{ErrHandler}({\bf var} \varfont{ChannelContext},\varfont{Verdict},\varfont{ExternalVals}):\\
\nudge if \varfont{Verdict} = ``Invalid'' then \\
\nudge\nudge \varfont{ExternalVals} $\gets$ ``Invalid'' \\
\nudge\nudge \varfont{ChannelContext.ChannelStatus} $\gets \langle
\mbox{Status ``closed"} \rangle$\\
\nudge Return \varfont{ExternalVals}
}
%
{
Given \varfont{ChannelContext}=\\[-0.5ex]
\nudge $\langle$ModeKey $K$, PRFKey $L$, ChannelStatus ``open'',\\
\nudge\nudge TokenList $ \mathit{TL}\rangle$
%Given $\varfont{Msg} =\langle V,H,C \rangle$

\medskip
Alg.\ \algfont{Deserialize}(\varfont{Msg}):\\
\nudge $V \gets \varfont{Msg}.V$; $H \gets \varfont{Msg}.H$; $C \gets \varfont{Msg}.C$\\
\nudge $\mathrm{sid} \gets H.\varfont{StreamID}$\\
\nudge Return $\left((V, H, \mathrm{sid}),C\right)$
%\nudge Return $\left((V, H),C\right)$

\medskip
Alg.\ \algfont{Decrypt}({\bf var} \varfont{ChannelContext},$(V,H,\mathrm{sid})$, $C$):\\
%\nudge $\varfont{ExternalVals} \gets H.\varfont{StreamID}$\\
\nudge Parse $C$ into tag $T$ and remainder $Z$\\ 
%\nudge $K \gets \varfont{ChannelContext.ModeKey}$\\
%\nudge $L \gets \varfont{ChannelContext.PRFKey}$\\
\nudge if $F_L(V,H,Z) \neq T$ then \varfont{InternalVals} $\gets$ ``TagFail''\\
\nudge $\mathrm{tid} \concat M \gets \calD_K^{V}(Z)$\\
\nudge if $\mathrm{tid}\not\in \mathit{TL}$ then\\
\nudge\nudge \varfont{InternalVals} $\gets \langle
\varfont{InternalVals}, \mbox{``BadToken"}, \mathrm{tid} \rangle$\\
\nudge \varfont{ExternalVals} $\gets \langle \mathrm{sid}, M \rangle$\\
\nudge Return (\varfont{ExternalVals},\varfont{InternalVals})

\medskip
Alg.\ \algfont{IsValid}(\varfont{InternalVals}):\\
\nudge \varfont{Verdict} $\gets$ ``Valid''\\
\nudge if ``TagFail'' $\in$ \varfont{InternalVals} then\\
\nudge\nudge \varfont{Verdict} $\gets$ ``Invalid''\\
\nudge if ``BadToken'' $\in$ \varfont{InternalVals} then\\
\nudge\nudge \varfont{Verdict} $\gets \langle
\varfont{Verdict},\mbox{``BadToken"} \rangle$\\
\nudge Return \varfont{Verdict}

\medskip
Alg.\ \algfont{ErrHandler}({\bf var} \varfont{ChannelContext},\varfont{Verdict},\varfont{ExternalVals}):\\
\nudge if ``Invalid'' $\in$ \varfont{Verdict} then \\
\nudge\nudge \varfont{ExternalVals} $\gets$ ``Invalid'' \\
\nudge\nudge \varfont{ChannelContext.ChannelStatus} $\gets \langle
\mbox{Status ``closed"} \rangle$\\
\nudge else if ``BadToken'' $\in$ \varfont{Verdict} then \\
\nudge\nudge \varfont{ExternalVals} $\gets$ ``Invalid'' \\
%\nudge\nudge \varfont{ChannelContext.ChannelStatus} $\gets$ none \\
\nudge Return \varfont{ExternalVals}
} 
%\tsnote{Is this actually useful for selling the ideas it is meant to sell?}
\todo{Tom: Describe the thing on the right in the caption.}
\caption{ {\bf
    Left:} Specification of Encrypt-then-MAC decryption over IV-based
  symmetric-key decryption $\calD$ and vector-input PRF~$F$. A tag-check failure results in the
suppression of all plaintext/errors output by $\calD^V_K(Z)$, and
causes a single error message to be exposed.  Also,
in the event of a tag-check failure, the error-handling algorithm
updates the \varfont{ChannelStatus} to
``closed''.  Note that this specification carries out decryption
$\calD^V_K(Z)$ even if there is a tag-check failure, thereby hiding a
potential timing side-channel. 
%
{\bf
  Right: } [...]
  }
\label{fig:EtM-aead}
\end{figure}

