\section{Misuse-Forgiving Primitives}

\paragraph{Prior Work.}
Towards the goal of building cryptography that is forgiving of misuse, prior
work by PI Shrimpton defined the notion of nonce-misuse-resistant authenticated
encryption~\cite{RS06}, and showed how to achieve it using standard
symmetric-key tools.  Rogaway had previously argued~\cite{xxx} that the
classical viewpoint that symmetric encryption schemes are randomized primitives
was misaligned with practice, and what we should be delivering are encryption
schemes that are deterministic, surfacing an explicit IV input. (This is an
early effort to close the gap between a theoretical primitive and its real-world
presentation.)  Moreover, to make schemes easier to use correctly, we should
target security when the IV is a non-repeating value ---~a nonce~--- rather than
demanding the IV be random.  In~\cite{RS06}, we sought to make these nonce-based
encryption schemes even easier to use, by designing them so that their security
guarantee degrades gracefully when the nonce IV repeats.  Nonce-misuse
resistance has been recognized as an important goal,  

Later work by PI Shrimpton~\cite{NRS} reconsidered the traditional wisdom about
building authenticated encryption (AE) via generic composition of an encryption
scheme and a MAC.   The seminal work by Bellare and Namprempre~\cite{BN} showed
that of the three classical compositions ---~encrypt-and-mac, mac-then-encrypt,
encrypt-then-mac~--- only the last is secure given any secure encryption scheme
and MAC.   This wisdom was heeded by an ISO standard, which would have been a
good thing, except that the ISO standard was mandating a nonce-based AE scheme,
whereas the Bellare-Namprempre results were about randomized AE.  As a result,
the ISO scheme was actually broken, despite the standard's (appaudible) efforts
to do what it seemed the crypto community told them.  Here again, the mismatch
between theoretical primitives and their real-world realization was problematic.
In~\cite{NRS}, we readdressed generic composition from the nonce-based
perspective, finding an interesting (albeit less simple) picture of what
compositions are (and are not) generically secure.  (Curiously, one of the
secure compositions that our work uncovered was the SIV-mode previously
published in~\cite{RS06} as the first nonce-misuse-resistant AE scheme.)

\tsnote{Hey Chris: what other types of misuse might be considered?
  Think about not just how one might misuse the inputs (say) to a
  primitive, but also how a primitive might be ``misused'' within an
  application, e.g. there may parallel communication channels, there
  may be metadata in the clear, there may be downgrading or other
  things that happen ``under the hood'' from the application's perspective, etc.}

\paragraph{Tasks.}
\begin{itemize}
\item {Follow-on work to hedging paper}\todo{Chris}
\item {Malleabilty models ideas?}
\item {Survey/user study to find out what things people don't
    understand, want to abuse, find annoying when using libraries/when
  trying to translate theory syntax into APIs}\todo{?}
\end{itemize}