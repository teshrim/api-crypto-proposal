% intro
\paragraph{Secure Channels.}
Even as the theory of cryptography grows to encompass newer and more exotic
challenges, providing a secure channel remains far and away its most mportant
task.
%
Today, TLS is deployed by ??\% of web servers and protects ??\% of
communications on the Internet. As such, this crucial code has been scrutinized
from top to bottom, including the crypto, the protocol specification,
implementations, and its deployment.\footnote{ See the \textsc{miTLS} project
for an overview: \url{https://www.mitls.org}.}
%
Academic papers on the (in)security of TLS fall braodly into three categories:
those focusing on the TLS \emph{handshake} (unilateral- or
mutually-authenticated key exchange)~\cite{BNF+14,others}, the \emph{record
layer} (the protocol encapsulating application data)~\cite{PRS11}, or the
composition of the two~\cite{JKSS12,KPW13}. This work will focus on the TLS
record layer.

Over the last decade or so, \emph{authenticated encryption with associated data}
(AEAD) has emerged as an important ingredient for constructing secure channels.
%
Its syntax is easy to reason about from a theoretical point of view, but is
robust enough to encompass most tasks that practitioners encounter. Moreover,
notions like \emph{nonce-misuse resistant}~\cite{RS07} and
\emph{robust}~\cite{HKR14} authenticated encryption make AEAD schemes difficult
to (inadvertently) use incorrectly. These properties make AEAD an attractive
tool for designing protocols; indeed, schemes like AES-GCM and
ChaCha20\mbox{-}Poly1305 are baked into the upcoming TLS 1.3
specification.\footnote{ See \url{https://tlswg.github.io/tls13-spec}.}
%
\cpnote{AFAIK AES-GCM is \emph{not} nonce-misuse resistant ... not sure about
ChaCha20. Are there other AEADs in the spec?}
%
It is tempting to think that AEAD, on its own, provides a secure channel. In
fact, for the thing to be useful for communicating over a network, we need to
specify a protocol in which it is used. The choice of protocol is driven largely
by engineering considerations, such as bandwidth constrains, computational
resources, and implementation complexity. However, the choice is also crucial
for security; time and time again, bugs in the \emph{protocol logic} have been
exploited to \emph{completely circumvent} the security provided by the
underlying crypto~\cite{BKN02}. These attacks point to a substantial gap between
crypto theory and the complexity of protocol logic.

We are by no means the first to point out this gap. It was recenlty observed by
Badertscher \etal~\cite{BMM+15} that the implicit goal of the TLS 1.3
record layer protocol is to construct an ``augmented'' secure channel (ASC), which
provide the capability of sending a message with two parts: the first being
private, and both parts being authenticated. They emphasize that classical
security notions, such as those for AEAD, ``do not capture in which contexts a
scheme satisyfying them can securely be used. They consider a specific attack
model and give certain capabilitiles to an adversry that tries to win some game,
but it is not \apriori clear which capababilities an adversary has in a
particular application, or even what her final goal is.'' \cite[pp. 2]{BMM+15}.
Nevertheless, this work (and others like it~\cite{BKN02,PRS11}) omit an important
detail.

\heading{Data is a stream \cite{FPMG15}.}
Providing a reliable (but not secure) channel between any two hosts on the Internet
is the job of the TCP/IP protocol stack. TCP in particular is designed so that
application developers need not worry about the details at all.
%
Conceptually, the API exports a stream-like functionality in which the sender
stuffs in any number of bytes and expects that they are read in order by the
sener.
%
In C, the sender sends data by invoking a function \codefont{send()}, which
takes as input its socket identifier, a buffer containing the data
\emph{fragment}, the fragment length (in bytes), and some flags; the output
indicates the number of bytes written to the channel, or an indication of
failure. The receiver invokes \codefont{recv()}, which takes as input its socket
identifier, a buffer, the maximum number of bytes to read from the stream, and
some flags. It returns the number of bytes read or an indication of failure.
%
Virtually all programming languages export a similar API for streaming data. The
alternative view (and the conventional wisdom of the cryptographic community) is
that channels provide transport of a sequence of discrete messages between two
parties. Of course, this is the true for datagram-based transport (UDP), but it
is not true for TCP.

The latter is ubiquitous for the delivery of web content; consequently,
implementations of the TLS record layer export a streaming API. In OpenSSL (the
most widely-deployed implementation of TLS), \codefont{SSL\_write()} is used to
write bytes to the stream and \codefont{SSL\_read()} is used to read bytes from
the stream. Although their syntax differes,\footnote{ See
\url{https://wiki.openssl.org/index.php/Manual:SSL_write(3)}.} these have the
same semantics as \codefont{send()} and \codefont{recv()} defined above. Thus,
the programmer can use this API just as she would use the lower-level
functionality for reliable transport, but with the added assurance that the
channel is secure.

Fischlin \etal~\cite{FPMG15} were the first (at least in our community) to
elucidate secure channels as a data stream. \cpnote{This is still a work in
progress.}




\ignore{
  \begin{hella}
  \end{cool}
}
